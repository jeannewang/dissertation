\documentclass[12pt]{ociamthesis}  % default square logo 
%\documentclass[12pt,beltcrest]{ociamthesis} % use old belt crest logo
%\documentclass[12pt,shieldcrest]{ociamthesis} % use older shield crest logo

%load any additional packages
\usepackage{amssymb}
\usepackage[margin=1in]{geometry} 
\usepackage{amsmath,amsthm,amssymb}
\usepackage{listings}% http://ctan.org/pkg/listings
\usepackage{graphicx}
\usepackage [autostyle, english = american]{csquotes}
\usepackage{soul}
\usepackage{qtree}
\usepackage{algorithm2e}
\usepackage{booktabs}
\usepackage{array}
\usepackage{hyperref}
\usepackage{color}
\usepackage[T1]{fontenc}
\usepackage[scaled]{beramono}

%input macros (i.e. write your own macros file called mymacros.tex 
%and uncomment the next line)
%\include{mymacros}
\MakeOuterQuote{"}
\DeclareMathOperator*{\argmax}{arg\,max}
\DeclareMathOperator*{\argmin}{arg\,min}
\newcommand{\ra}[1]{\renewcommand{\arraystretch}{#1}}
\newcolumntype{L}[1]{>{\raggedright\let\newline\\\arraybackslash\hspace{0pt}}m{#1}}
\newcolumntype{C}[1]{>{\centering\let\newline\\\arraybackslash\hspace{0pt}}m{#1}}
\newcolumntype{R}[1]{>{\raggedleft\let\newline\\\arraybackslash\hspace{0pt}}m{#1}}


\definecolor{mygreen}{rgb}{0,0.5,0}
\definecolor{mygray}{rgb}{0.5,0.5,0.5}
\definecolor{mymauve}{rgb}{0.9,0,0}

\definecolor{bluekeywords}{rgb}{0.13,0.13,1}

\lstset{ %
  language=C++,
  backgroundcolor=\color{white},   % choose the background color; you must add \usepackage{color} or \usepackage{xcolor}
  basicstyle=\ttfamily\scriptsize,        % the size of the fonts that are used for the code
  breakatwhitespace=false,         % sets if automatic breaks should only happen at whitespace
  breaklines=true,                 % sets automatic line breaking
  captionpos=b,                    % sets the caption-position to bottom
  commentstyle=\color{mygreen},    % comment style
  keywordstyle=\color{bluekeywords}\bfseries,
  deletekeywords={...},            % if you want to delete keywords from the given language
  escapeinside={\%*}{*)},          % if you want to add LaTeX within your code
  extendedchars=true,              % lets you use non-ASCII characters; for 8-bits encodings only, does not work with UTF-8
  %frame=single,                    % adds a frame around the code
  keepspaces=true,                 % keeps spaces in text, useful for keeping indentation of code (possibly needs columns=flexible)
  keywordstyle=\color{blue},       % keyword style
  morekeywords={*,...},            % if you want to add more keywords to the set
  numbers=left,                    % where to put the line-numbers; possible values are (none, left, right)
  numbersep=5pt,                   % how far the line-numbers are from the code
  numberstyle=\tiny\color{mygray}, % the style that is used for the line-numbers
  rulecolor=\color{black},         % if not set, the frame-color may be changed on line-breaks within not-black text (e.g. comments (green here))
  showspaces=false,                % show spaces everywhere adding particular underscores; it overrides 'showstringspaces'
  showstringspaces=false,          % underline spaces within strings only
  showtabs=false,                  % show tabs within strings adding particular underscores
  stepnumber=2,                    % the step between two line-numbers. If it's 1, each line will be numbered
  stringstyle=\color{mymauve},     % string literal style
  tabsize=2,                       % sets default tabsize to 2 spaces
  caption=train\_hlbl.cc                   % show the filename of files
}

\title{An Exploration of Novel Trees for the Hierarchical Log Bilinear Language Model}

\author{Jeanne Wang}             %your name
\college{Exeter College}  %your college

%\renewcommand{\submittedtext}{change the default text here if needed}
\degree{Master of Science in Computer Science}     %the degree
\degreedate{September 2013}         %the degree date

%end the preamble and start the document
\begin{document}

%this baselineskip gives sufficient line spacing for an examiner to easily
%markup the thesis with comments
\baselineskip=18pt plus1pt

%set the number of sectioning levels that get number and appear in the contents
\setcounter{secnumdepth}{3}
\setcounter{tocdepth}{3}


\maketitle                  % create a title page from the preamble info
%\begin{dedication}
This thesis is dedicated to\\
 someone\\
for some special reason\\
\end{dedication}        % include a dedication.tex file
%\begin{acknowledgements}

I would like to thank my supervisor Phil Blunsom for his help and insightful comments.

Additionally, thank you to Andriy Mnih for answering the questions of a puzzled masters student.

Thank you to Tom{\'a}\v{s} Ko\v{c}isk{\'y} for listening to me rant, for sitting down with me while calculating the gradients and for editing my drafts. 

My gratitude also goes to Adela Lica for editing my dissertation.

I would lastly like to thank all of the friends that made my experience at Oxford so memorable and enjoyable.

\end{acknowledgements}   % include an acknowledgements.tex file
%\begin{abstract}

Statistical language models are of great importance to natural language processing tasks. A major problem with current state-of-the-art, neural-based language models is that they tend to be slow to train and test. Neural models can easily take weeks to train on large corpora. 

This dissertation proposes to use hierarchical decompositions of a particular neural model, the log-bilinear model, in order to speed up training and testing times. Using a hierarchical log-bilinear model, training and testing times can be logarithmically sped up. This dissertation focuses on the new combination of the hierarchical log bilinear model and three forms of hierarchical structure: Huffman Trees, Brown Cluster trees, and Recursive ADAPTIVE trees. Additionally, this dissertation explores a new value for model initialization.

We assess the models with a series of experiments that show the pros and cons of each method. A hierarchical log-bilinear model with an ADAPTIVE tree trained on semantic/syntactic word embeddings and with proper initialization is shown to perform $3.89\times$ faster than the factored log-bilinear model with an acceptable entropy loss of $1.93\%$ on the British National Corpus dataset.

\end{abstract}          % include the abstract

\begin{romanpages}          % start roman page numbering
\tableofcontents            % generate and include a table of contents
\listoffigures              % generate and include a list of figures
\end{romanpages}            % end roman page numbering

%now include the files of latex for each of the chapters etc

\chapter{Introduction}
\paragraph{}
In the field of computational linguistics, statistical language modeling is a cornerstone task. The use of language models  is universal in applications such as machine translation and speech recognition.  Improving upon language models can improve the performance of these important applications and future applications that involve language models.
\section{Motivation}
\paragraph{}
The state-of-the-art language models are all neural-based models. In spite of this, n-gram models are still incredibly prevalent and widely used. Neural-based models are unattractive to use due to their long training and testing times. The training of a large neural-network based model can take weeks, as opposed to the minutes an n-gram model takes. While neural networks may never be as fast as n-gram models to train and test, these models can become much more attractive with shorter training and testing times. If a user only had to wait hours for his model instead of weeks, neural models become a much more persuasive option.
\paragraph{}
Additionally, given the large amount of buzz and research around \emph{deep-learning}, neural models are only going to get better in the future. It becomes paramount to find ways to speed up these models for both research and commercial usage.
\paragraph{}
One way to speed up these models is to apply a hierarchical structure to the neural model. This dissertation will explore using such a hierarchical model and new forms of hierarchy.

\section{Objectives}
\paragraph{}
One simple neural-based model with good performance is the log-bilinear model \cite{MnihHinton2007}. I would like to apply a hierarchical structure to it in the form of the hierarchical log-bilinear model \cite{MnihHinton2009}. The hierarchical structure, or tree, can be interchanged freely. I would like to explore various trees to improve the model performance. The objective of this project is to explore novel trees with the hierarchical log-bilinear model in order to find a tree that is fast to train, test and create and also performs well in comparison to the standard log-bilinear model.

\section{Contributions}
\paragraph{}
In this dissertation, I explore the combination of three trees with the HLBL model: the Huffman tree, the Brown Cluster tree and the Recursive ADAPTIVE tree. These three trees have not been used before with the HLBL model. I find that the Brown Cluster tree is the best performing tree in terms of perplexity of the three, but is very slow to create. The Huffman tree on the other hand is fast to create, but does not perform very well in terms of perplexity. The Recursive ADAPTIVE tree is in the middle of the pack; it is relatively slow to create but has better performance than the Huffman tree. To improve the performance of the ADAPTIVE tree model, I use word2vec representations to initialize the context word representation parameters. This proves to be a good combination and results in a model that is both fast and accurate when applied to large corpora. The new model is $3.89\times$ faster than the factored log-bilinear model but is 1.93\% worse in terms of entropy.

\section{Outline}
\paragraph{}
In Chapter 2 of this dissertation I will cover language models including the n-gram model, maximum entropy model, neural probabilistic model, feed-forward neural network model, recurrent neural network model and the log-bilinear model. Additionally, I will describe extensions to these models and model evaluation techniques.

\paragraph{}
In Chapter 3, I will cover speed-up techniques for language models and describe why I have decided to focus on using a hierarchical method. These speed-up techniques include class-based methods, hierarchical-based methods, gradient approximation methods, vocabulary truncation, diagonalizing context matrices, and also switching the training order.

\paragraph{}
In Chapter 4, I will explain my proposed method including the model definition and the tree definitions. The proposed model is the hierarchical log-bilinear model and the trees that will be explored are the Huffman tree, the Brown Cluster tree, and the Recursive ADAPTIVE tree.

\paragraph{}
In Chapter 5, I describe the engineering tasks and tips and tricks for implementing the model. These include model initialization, tree building, gradient checking and variable caching.

\paragraph{}
In Chapter 6, I describe the datasets used, the experiments undertaken, the reasoning behind each experiment and also state their results. The results are then compared and analyzed. 

\paragraph{}
In Chapter 7, I derive conclusions for the project and also recommend directions for future work.

\paragraph{}
Lastly, I include three appendices. Appendix A gives a derivation of the gradients used in training the model. Appendix B holds visualizations of the word and node embeddings learned by the models. Appendix C holds a copy of the most relevant parts of my C++ implementation of the model.


\chapter{Background}
\paragraph{}
Statistical language models have had much academic and commercial attention for the last few decades. Academic research has traversed various flavors of language models including n-gram models, maximum entropy models, neural probabilistic models, log-bilinear models, feed-forward and recurrent neural network models. This chapter will cover the above models.
\paragraph{}
Language models have real-world usages in well-known applications such as speech recognition and machine translation. The most notable commercial example of language model usage is Google Translate. Given the sheer amount of textual data that is available, statistical language models are the only practical way to deal with language modeling. Additionally, since the amount of data is growing,  statistical language models retrained on new data can become even more useful and more accurate over time.

\section{Statistical Language Models}
\paragraph{}
A statistical language model is a probabilistic model that computes the joint probability of a particular sequence of  $m$ words.  
\begin{align}
P(w_1, \dots ,w_m)
\end{align}
The joint probability can be rewritten as the product of the conditional probabilities of an upcoming word given the previous $m-1$ words.
\begin{align}
P(w_1, \dots ,w_m) = \prod_{i=1}^m P(w_i | w_1,\dots, w_{i-1})
\end{align}
We can therefore think of an individual word $w_m$'s probability as a conditional probability.
\begin{align}
P(w_m | w_1,\dots, w_{m-1})
\end{align}
In this dissertation, I will generally define language models in terms of individual word probabilities.
\paragraph{}
Language modeling is typically applied to natural languages such as English, French, Chinese, etc. There is a large bias towards certain languages, especially English in terms of corpora and subsequently research. Statistical language models have benefited greatly from the proliferation and widespread availability of large text corpora on the Internet.  Having larger amounts of data is particularly useful in statistical language modeling since model parameters are trained more accurately with larger training sets. 

\section{Language Model Usages}
\paragraph{}
Language models are an integral part of many computational linguistic applications, most notably speech recognition and machine translation. They have also been important in information retrieval tasks, spelling correction, and optical character recognition. Language models provide a priori knowledge of what a particular language looks like. Improving language models used in these applications, can improve the overall performance of these applications \cite{Jurafsky2009}. 

\section{Training and Testing sets}
\paragraph{}
Statistical language models are generally trained and tested on different subsets of the same data. Both training and testing sets are drawn from the same distribution. If the model is trained on the training set without overfitting, it should generalize well to the test set. In practice, a third development set is often used as a substitute for the test set during development. This is to prevent bias in hand-tweaked parameters when finally testing on the test set. Often, the full corpus is divided randomly into 80\% training set, 10\% development set, and 10\% test set.

\section{N-gram Models}
\paragraph{}
The \emph{n-gram} model is a widely used language model and forms the basis for more advanced language models. The n-gram model is essentially a look-up table of word co-occurance statistics \cite{Jurafsky2009}. Due to its look-up table nature, the n-gram model is bad at generalizing to test data absent from the training data. The $n$ in n-gram comes from the $n -1$ word context that the n-gram model considers to predict the word probability of the $n$th word. A unigram model will use zero words for its history, a bigram model will use a single word for its history, a 3-gram model will use two words for its history, etc.  Even though n-gram language models are not complicated, they are very effective when training and testing sets are similar. N-gram models are the de facto standard language model. More complicated models only marginally best n-gram performance and are much more computationally expensive to train/test \cite{Mikolov2012}. 
In an n-gram model, the probability of observing a sequence $w_1, \dots, w_m$ is approximated using the Markov assumption:
\begin{align}
P(w_1,\dots,w_m) = \prod^{m}_{i=1} P(w_i|w_1,\dots, w_{i-1}) \approx  \prod^{m}_{i=1} P(w_i | w_{i-(n-1)},\dots, w_{i-1}) 
\end{align}
The maximum likelihood estimation of a word probability is given by:
\begin{align}
P(w_i | w_{i-(n-1)},\dots, w_{i-1}) = \frac{count(w_{i-(n-1)},\dots,w_{i-1},w_i)}{count(w_{i-(n-1)},\dots,w_{i-1})} 
\end{align}

\subsection{Markov Assumption}
\paragraph{}
The \emph{Markov assumption} used in language models states that instead of using the entire previous history of words to compute the probability of the next word, we can use simply the $n$ previous words \cite{Jurafsky2009}. The intuition is that nearby words are more statistically significant than words that are farther away. Also, the order of the last $n$-words is taken into account. While this assumption is false, in practice it seems to work well. N-gram models and all of the subsequent language models that are discussed use the Markov assumption.
\subsection{Curse of Dimensionality}
\paragraph{}
The \emph{curse of dimensionality} refers to the increasing difficulty of searching though high dimensional spaces as the number of dimensions grows. The curse of dimensionality diminishes the benefit of contexts larger than five for n-grams. Unfortunately this means that n-grams cannot express long-distance relationships or long-distance dependencies in texts. Data becomes sparse as we increase the number of dimensions. Even over millions of examples in large corpora, it is rare to see the same long word sequence repeated (where long is 6+ words). Again due to the look-up nature of the n-gram model, if exact test data sequences are not seen in training data then n-gram models perform poorly.
\subsection{Smoothing}
\paragraph{}
N-gram models deal with unseen word sequences (often due to the curse of dimensionality, or small corpora) using smoothing. Smoothing "\dots redistribut[es] probabilities between seen and unseen (zero-frequency) events, by
exploiting the fact that some estimates, mostly those based on single observations, are greatly over-estimated" \cite[pg. 16]{Mikolov2012}.
\subsubsection{Laplacian Smoothing}
\paragraph{}
The simplest form of smoothing is \emph{Laplacian smoothing}, and is also known as \emph{add-one smoothing}~\cite{Jurafsky2009}. With this type of smoothing, all sequence counts are increased by one. This means that all sequences will have a non-zero probability. The probability of a word using Laplacian smoothing is given by:
\begin{align}
P(w_i | w_{i-(n-1)},\dots, w_{i-1}) = \frac{count(w_{i-(n-1)},\dots,w_{i-1},w_i)+1}{count(w_{i-(n-1)},\dots,w_{i-1})+|V|}
\end{align}
where $|V|$ is the size of the vocabulary. While this does solve the issue of zero-frequency elements, Laplacian smoothing gives bad estimates. This is because Laplacian smoothing does not take advantage of any prior information about the word probability distribution.
\subsubsection{Advanced Smoothing Techniques}
\paragraph{}
 Other smoothing techniques that have been applied to n-gram models include: \emph{Good-Turing models, Katz back-off models, interpolated models}, and \emph{modified Kneser-Ney models}. Modified Kneser-Ney is generally the preferred smoothing method due to its good performance. These smoothing techniques use knowledge of rarely seen words to estimate counts of unseen words. The Good-Turning approach modifies the counts of word sequences to be more similar to word sequences seen more frequently. The Katz back-off model backs-off to lower order n-gram models for zero-frequency word sequences. Interpolated models interpolate between different order n-gram models. Modified Kneser-Ney smoothing combines back-off smoothing and model interpolation \cite{Jurafsky2009}.

\section{Maximum Entropy  Models}
\paragraph{}
\emph{Maximum entropy} models (MaxEnt), also known as \emph{multinomial logistic regression models}, are linear models often used for classification. They are good at taking in large numbers of different types of features. These features are often hand selected. When used as a language model, MaxEnt models typically use n-grams or \emph{skip-grams} as features \cite{Mikolov2012}. A skip-gram is an adaptation of n-grams where a number of skipped words are allowed when constructing the n-gram. For a more formal definition of skip-grams see \cite{Guthrie2006}. A n-gram can be thought of as a skip-gram with 0 skips. In a MaxEnt model, each feature gives rise to a constraint that the model must follow. A valid probability distribution produced by an MaxEnt model must conform to all of its constraints. Often there is more than one valid probability distribution, so the distribution with the highest entropy is chosen, generally, using a maximum likelihood estimator. The original MaxEnt language model is described in \cite{Rosenfeld1994}.
MaxEnt models have the form:
\begin{align}
&P(w_i | w_{i-(n-1)},\dots, w_{i-1}) = \frac{e^{\sum_i \lambda_i f_i(w_i | w_{i-(n-1)},\dots, w_{i-1})}}{Z(w_{i-(n-1)},
\dots, w_{i-1})} \label{eq:maxent}
\\
&Z(w_{i-(n-1)},\dots, w_{i-1}) = \sum_{w_i \in V} e^{\sum_i \lambda_i f_i(w_i | w_{i-(n-1)},\dots, w_{i-1})} \nonumber
\end{align}
In this formulation, $f_i(w_i | w_{i-(n-1)})$ is a feature, $\lambda_i$ is a weight for that feature, and $Z$ is the normalizing constant for a given context.
\paragraph{}
MaxEnt models perform competitively with other top language models. Chen et al., used a regularized, class-based MaxEnt model, the M model, to achieve a 4\% reduction in perplexity over a 4-gram modified Kneser-Ney model \cite{Chen2009}. To put this in context, a 4-gram or 5-gram Kneser-Ney model is the standard n-gram used for language model comparisons.
\paragraph{}

MaxEnt models can be slow to train and test due to their soft-max formulation which requires normalizing over all words in the vocabulary. A \emph{soft-max} output function produces outputs between 0 and 1, that sum up to 1, and that can be interpreted as probabilities. 

\section{Neural Network Based Models} \label{sec:nplm}
\paragraph{}
Neural network based models represent words as real-valued, low-dimensional vectors. Neural networks learn these word representations with the hope that these learned vectors capture non-trivial features of the word such as semantic and syntactical patterns.
\paragraph{}
Neural network based models generalize better than n-grams because neural network based models are more flexible with their history representations. To perform well, n-grams require exact sequences to be in both the training and test sets. Neural network based models, on the other hand, can generalize and only need similar sequences in the training and test data to ensure good performance.

\subsection{Neural Probabilistic Model}
\paragraph{}
The first rigorous application of a neural network to a statistical language model was done by Yoshua Bengio \cite{Bengio2003}. Neural network based models are explored more in \cite{Collobert2008}, and \cite{HuangEtAl2012}. In Bengio's words, his \emph{neural probabilistic model} (NPLM), "learn[s] a distributed representation for words which allows each training sentence to inform the model about an exponential number of semantically neighboring sentences" \cite[pg. 1137]{Bengio2003}. The main benefit of NPLM models is that the learned distributed \emph{word feature vectors}, also known as \emph{word embeddings}, generalize to contexts larger than the nearest $n$ words and also learn representations that account for the semantic and syntactical similarity between words. As a consequence, similar words have similar word feature vectors, at least, along some dimensions. Word feature vectors have low dimensionality typically between 30 to 100 dimensions. The dimensionality should be much smaller than the size of the vocabulary. In the NPLM model, the word feature vectors and the model are jointly learned. The model is trained using sequences of word feature vectors that correspond to sentences/sequences of words in the training set. The neural network then learns the probability distribution of the next word given the context features by maximizing the log-likelihood of the training data. 
The feed-forward NPLM model is given by:

\begin{align}
&P(w_i | w_{i-(n-1)},\dots, w_{i-1}) = g(x_i) = \frac {e^{b+Wx_i+U \tanh(d+Hx_i)}} {\sum_i e^{b+Wx_i+U \tanh(d+Hx_i)}} \label{FFNN}
\\
&x_i=\left(C(w_{i-(n-1)}), \dots, C(w_{i-1}) \right) \nonumber
\end{align}
where $|V|$ is the size of the vocabulary and $m$ is the number of dimensions of the distributed vector representation. The $C(w_{n-1}), \dots, C(w_{t-n+1})$ terms are context matrices with an individual $C_i$ matrix for each of $n$ context indices where $n$ is normally between 2 to 10. Each $C_i$ matrix is of size $|V|\times m$ and holds the distributed vector representations for each word, per context index. The $g$ function is a neural network, either a parameterized feed-forward or recurrent neural network. 
"The free parameters of the model are the output biases b (with $|V|$ elements), the hidden layer biases $d$ (with $h$ elements), the hidden-to-output weights $U$ (a $|V| \times h$ matrix), the word features to output weights $W$ (a $|V| \times (n-1)m$ matrix), the hidden layer weights $H$ (a $h \times (n-1)m$ matrix), and the word features $C$ (a $|V| \times m$ matrix)" \cite[pg. 1143]{Bengio2003}.
\paragraph{}
Note the non-linearity introduced by the hyperbolic tangent in the model. This non-linearity is a key feature of neural network models.
\paragraph{}
The model parameters are trained by stochastic gradient ascent using \emph{backpropagation} to maximize the regularized log likelihood. Backpropagation is described in Section \ref{sec:FFNNRNN}.
\paragraph{}
The NPLM model generalizes well as the model maps similar input vectors to similar outputs. Unknown words that occur in similar contexts as training words will have similar distributed vector representations. The unknown words will therefore have probabilities close to the known words.  Due to this, the NPLM model performs very well in comparison to other top language models. Bengio's model performed 24\% better in terms of perplexity on the Brown corpus and 8\% better on the AP News corpus compared to a 5-gram modified Kneser-Ney model.
\paragraph{}
Unfortunately, NPLM models are slow to train and test, again due to the normalization over all words in the vocabulary needed in the soft-max function. Bengio's NPLM model took over 3 weeks to train 5 epochs using 40 CPUs on the AP News corpus which has a vocabulary size of $|V|=$16,383, which is small in comparison to modern corpuses.

\subsection{Feed Forward and Recurrent Neural Network Models} \label{sec:FFNNRNN}
\paragraph{}
The \emph{feed forward neural network model} (FFNN) was originally proposed by Yoshua Bengio in \cite{Bengio2003}. The FFNN is the neural network used in the NPLM. Tom{\'a}\v{s} Mikolov proposes a simpler architecture in \cite{Mikolov2009}.

\paragraph{}
The FFNN model is formulated as above in Equation \ref{FFNN}. The FFNN model takes \emph{one-hot} word vectors (only the dimension corresponding to the word is 1, all other dimensions are 0) and projects the vectors down to a much smaller hidden layer which clusters words together. The hidden layer is then introduced to a non-linear activation function such as a hyperbolic tangent or sigmoid function. Lastly, the hidden layer is projected back up to the word vector size and run though a soft-max function to produce word probabilities. The FFNN model is trained using \emph{backpropagation}. Backpropagation is a two-step algorithm with a forward and backward pass. In the forward pass, the algorithm uses the parameters to compute the outputs. On the backward pass, the output errors are propagated backwards into the parameters with respect to the amount of error each parameter is responsible for. Backpropagation is described in \cite{Bengio2003}.

\paragraph{}
The \emph{recurrent neural network model} (RNN) is similar to a FFNN model except that it uses an essentially infinite history. A RNN model learns a representation of all previous history. The FFNN model, on the other hand, is only informed about the last $n-1$ words. The model is called a recurrent neural network, because it feeds its own output back in as an input, like a recursive function. The RNN model is trained by maximizing the log likelihood using \emph{backpropagation though time}. Backpropagation though time is a training algorithm similar to backpropagation. Instead of having a RNN with one hidden layer used $n$ times, backpropagation through time unfolds the RNN model into a FFNN model with $n$ hidden layers, and then applies straightforward backpropagation. The backpropagation through time procedure is described in \cite{Mikolov2012}.

\begin{figure}
\centering
\includegraphics[height=300px,trim=100 380 100 30,clip=true]{./images/rnn_pdf.pdf}
\caption{Simple Recurrent Network \cite[pg. 34]{Mikolov2012}.}
\label{fig:RNNfigure}
\end{figure}

The RNN model is formulated:
\begin{align}
&s_j(t) = \sigma \left( \sum_i w_i(t) u_{ji} + \sum_l s_l (t-1) w_{jl} \right)
\\
&y_k(t) = softmax \left( \sum_j s_j(t) v_{kj} \right)
\\ 
&\sigma(x) = \frac{1}{1+e^{-x}} \nonumber
\\ 
&softmax(x_m) = \frac{e^{x_m} }{ \sum_i e^{x_i} }  \nonumber
\end{align}
where $w(t)$ is a one-hot representation of the word, $U$ and $W$ hold the matrix parameters between the input and hidden layer, and $V$ is the matrix holding parameters between the hidden and output layer. Refer to Figure \ref{fig:RNNfigure}.

\paragraph{}
The class-based RNN interpolated with the Recurrent Neural Network Maximum Entropy language model by Mikolov \cite{Mikolov2012} currently has state-of-the-art performance. The dynamically evaluated interpolated model has 44.19\% reduction in perplexity and 11.8\% reduction in entropy compared to a 5-gram modified Kneser-Ney model on the Penn Treebank.
\paragraph{}
Mikolov states that "[r]oughly speaking, we can reduce training times from many weeks to a few days by using the novel RNN architecture" on $400$ million tokens and with $|V|=$82,000 \cite[pg. 93]{Mikolov2012}. 

\subsection{Log-Bilinear Model} \label{sec:lbl}
\paragraph{}
The \emph{log-bilinear model} (LBL) is a simple neural network based model. Andriy Mnih and Geoffrey Hinton introduce the LBL model in \cite{MnihHinton2007}. Unlike a typical neural model, the LBL model has no non-linearities introduced except for the soft-max output function. By having no non-linearities, the LBL model is much easier to understand. The target word representation can always be linearly decomposed into its constituent context word representations. Unfortunately, this means the LBL model cannot capture non-linear phenomena in its representations. Like other neural models, the LBL model has the benefit of learning distributed vector word representations for both the input and output words.  The LBL model computes a prediction vector for the next word from a linear combination of the context word vectors. This prediction vector is then compared to the actual target word vector using an unnormalized cosine similarity function. The measure of similarity between these vectors is then run through the soft-max function to give a conditional word probability.

The LBL model is formulated:
\begin{align}
P(w_i | w_{i-(n-1)},\dots, w_{i-1})  =& \frac{ \exp( \hat{r}^T q_{w_i} +b_{w_i}) } { \sum_j \exp( \hat{r}^T q_{w_j} +b_{w_j})} \label{eq:LBL}
\\ 
\hat{r} =& \sum_{i=1}^{n-1} C_i r_{w_i} \nonumber
\end{align} 
where $r_{w_i}$ is the vector representation for context word $w_i$ and $q_{w_i}$ is the vector representation for the target word $w_i$. $C_i$ is the matrix that corresponds to the interaction between the $i$th context word and the target word. $b_{w_i}$ is a bias term for word $w_i$. $\hat{r}$ can be thought of as a prediction vector for $w_i$. Note that there are separate representations $r_{w_i}$, and $q_{w_i}$ for context words and target words respectively, even if the actual word is the same.

\paragraph{}
The model is trained using stochastic gradient ascent to maximize the log likelihood. The training is described in \cite{MnihHinton2007}.
\paragraph{}
Mnih and Hinton report a 5.3\% perplexity improvement with respect to the back off Kneser-Ney 5-gram model on the AP News dataset with 14 million training tokens and $|V|$=17,964. The LBL model is relatively fast for a neural model as it has no non-linearities to compute, though again suffers from the soft-max normalization. 

\section {Extensions to Language Models}
\paragraph{}
The language models described above can be improved with a few extensions.

\subsection{Interpolated Models}
\paragraph{}
Combining models can result in a better model. An interpolated model linearly interpolates word probabilities generated by two or more models trained on the same data. The weight on each constituent model is hand-tweaked or learned. The interpolated model benefits from all the constituent models that may have learned different features of the language.
\subsection{Cached Models}
\paragraph{}
Words that are rare in a language may be frequent in an individual document. For example, "kettlebell" may be a rare word in the English language but may be common in an exercise article. To deal with this phenomenon, cache models are used. The original model is proposed in \cite{Kuhn1990}. The basic idea is to train two models, a normal language model, and a second model on a small portion of the training/test set that corresponds to the document in question. The~two models are then interpolated to get a cache model. The intuition is that rare words that are common in the document will be cached in the second model, while the first model will deal with the rest of the language.

\section {Evaluation of Language Models}
\paragraph{}
Given different models, we need a way to compare their performance. One standard comparison score is model perplexity. In this dissertation, I will use perplexity to evaluate my model. Note that to compare perplexities, the models must be trained and tested on the same data.
\subsection{Perplexity} \label{sec:perplexity}
\paragraph{}
\emph{Perplexity} (PP) can be thought of as a measure of how "perplexed" the model is by a test set.
Perplexity is given by:
\begin{align}
PP(w_i | h_i)=e^{- \frac{1}{K} \sum_{i=1}^K \ln( P(w_i | h_i) ) } \label{eq:perplexity}
\\ h_i = ( w_{i-(n-1)},\dots, w_{i-1} ) \nonumber
\end{align}
where $K$ is the number of testing examples. 
\paragraph{}
Perplexity is $e$ raised to the average \emph{Shannon entropy} of the probability distribution. Entropy can be thought of as the average number of base $e$ units required to encode the expected value of the information contained in the probability distribution. A better model will be less "perplexed" when it sees new data, and so will have a lower perplexity.  Perplexity is normally calculated with respect to word sequences from a testing set. 
\paragraph{}
Perplexity is an intrinsic evaluation metric that measures the performance of the model alone. Language models are often used in larger applications, such as machine translation, where the end-to-end performance of the application is important. Perplexity does not give a measure of end-to-end performance, however perplexity improvement does positively correlate to end-to-end improvement \cite{Jurafsky2009}.
\paragraph{}
It is convenient to use perplexity instead of entropy as a measure since entropy is generally measured in base $2$. By using perplexity, we can compare models even when entropy is measured in a different base. On the other hand, Mikolov makes an important point about entropy versus perplexity when it comes to measuring improvement \cite{Mikolov2012}. A percent of perplexity reduction does not correlate to a percent of entropy reduction. This is due to the exponential relationship between perplexity and entropy. Therefore, if comparing reduction in scores, it is better to use entropy or a combination of entropy and perplexity. Unfortunately, many research papers still only use reduction in perplexity as a measure.

\chapter{Related Work}
\paragraph{}
The biggest limitation of neural based language models is their slow training and testing. Subsequently, there has been a lot of work in the field dedicated to speeding up neural-based models. Methods that have been used to speed up such models include creating class-based models, creating tree-based models, using gradient approximation, truncating the vocabulary, using diagonal context matrices, and changing the training order.
\paragraph{}
The slowest part of neural models tends to be the soft-max normalization used in the output function, due to the normalization over a large vocabulary. Large vocabulary sizes also affect the time to calculate the expectation component of the objective function gradient. In addition to vocabulary size ($|V|$), other factors that affect training times include: number of training iterations ($I$), number of training tokens ($W$), number of hidden layers (H), context-size~($N$), and word representation dimensionality ($D$). Neural network model training has time complexity $O\left( I \times W \times \left( (N-1) \times D \times H +  |V| \times H \right) \right)$ \cite{Mikolov2012}. The vocabulary size and number of training tokens are by far the largest contributors to the training time. Given that we would like to train on as many examples as possible, $W$ should not be reduced. This leaves~$|V|$ as the main component we should focus on reducing. Note, that the training complexity is linear in the size of the vocabulary.
The methods described below try to speed up one or more of the contributors to the training complexity. Though these methods are explored for a single model, the ideas can be applied to speed up other models.
\section{Class-based Model}
\paragraph{}
\begin{figure}
\Tree [. [.noun [.animal {cat}  {rabbit} {dog} {\dots} ] [.{household object} {vacuum} {broom} {\dots}  ] [.{\dots} ] ]   {\dots}  ]
\caption{Simple 3-level class based model}
\label{fig:3class}
\end{figure}

Joshua Goodman proposed a class-based method for speeding up training and testing of MaxEnt models in \cite{Goodman2001}. In a class-based language model, there are two steps. The first step is to predict the class of the word and the second step is to predict the word. Each word is assigned to a single class. 
The probability of a word is given by:
\begin{align}
P(w_i | h_i)  =&  P(class(w) | h_i) \times P(w|  h_i, class(w))
\\ h_i =& ( w_{i-(n-1)},\dots, w_{i-1} ) \nonumber
\end{align}
The constituent probabilities $P\left(class(w) | h_i \right)$ and $P \left(w|  h_i, class(w) \right)$ are predicted using two separate MaxEnt models. The MaxEnt formulation is given by Equation \ref{eq:maxent}. The MaxEnt model uses a soft-max output function which normalizes over all words in the vocabulary.
\paragraph{}
The idea is to limit the number of possible outputs, which is the same as limiting the effective size of the vocabulary. By first selecting a class, the word probability can be normalized over the class size instead of the vocabulary size. If the words are distributed uniformly across 100 classes, then the first normalization will be over 100 (for the classes) and the second normalization will occur over $\frac{1}{100}$th the vocabulary size (for the word). This can drastically reduce training times. For example, if there are 10,000 words in the vocabulary, there will be two normalizations over 100, with time proportional to 200. The normal soft-max output would normalize over the entire vocabulary in time proportional to 10,000. In this case, there is a 98\% reduction in training time. In general there will be an order $O(\sqrt{|V|})$ speedup.
\paragraph{}
Goodman also suggested using additional prediction levels, though he found that more than three levels did not lead to large improvements. In a three level system such as Figure \ref{fig:3class}, for the word \emph{dog}, a model might first predict the super-class \emph{noun}, then predict the class \emph{animal}, and finally predict the word \emph{dog}. The classes and super-classes are generally semantic or lexical classes and are generated by clustering based on some similarity metric.
\subsection{Frequency-binned classes} \label{sec:frequencybinning}
\paragraph{}
Instead of using classes that are semantically or lexically motivated, frequency-binned classes can also be used. Words can be assigned to classes based on their relative frequency. This would still have the speed advantage of the class-based structure but would remove the overhead of creating "smart" classes. Though frequency binning is very easy to implement, the loss of "smart" classes, does have negative effects on the model performance. With a semantic class based system, words in the same class will be semantically similar. Therefore if the wrong word is predicted for a particular class, at least the predicted word will be similar to the correct word. With frequency binning, an incorrectly selected word could be very semantically different. Even so, Mikolov in \cite{Mikolov2012} claims frequency-binned classes work well. 

\section{Tree-based Model}
\paragraph{}
Another method to speed up the training and testing of neural models language models is to arrange the vocabulary into a decision tree structure. This way, each word becomes a series of decisions through the tree. The number of decisions is order $O(\log|V|)$.

\subsection{Hierarchical Neural Network Language Model}
\paragraph{}
The \emph{hierarchical neural probabilistic language model} (HNPLM) is introduced by Frederic Morin and Yoshua Bengio in \cite{MorinBengio2005}. This idea is an extension of the class-based method proposed by Goodman. Compared to a class-based method,  a hierarchical method increases the number of prediction class levels, and also limits the number of elements in each class. If we take this idea almost to the limit, we end up with $\log_2(|V|)$ class levels and 2 elements in each class. As described in \cite{MorinBengio2005}, the hierarchy is represented by a balanced binary tree where each node represents a class with two constituents.  A word probability is then the product of all conditional class probabilities on the path through the tree to the word node as below:
\begin{align}
&P(w_i | w_{i-(n-1)},\dots, w_{i-1})  =  \prod_{j=1}^{k} P(b_j(w_i)|node_j ,  w_{i-(n-1)},\dots, w_{i-1})
\\
&P(b_j(w_i)=1|node_j,  w_{i-(n-1)},\dots, w_{i-1}) =  \sigma(\alpha_{node_j} + \beta' \tanh(c+ Wx+ UN_{node_j})) \label{eq:hplm_sig}
\\
&node_j =b_1(w_i),\dots,b_{j-1}(w_i) \nonumber
\\ 
&\sigma(x) = \frac{1}{1+\exp(-x)} \nonumber
\end{align}
Where $b_j(w_i)$ is the binary decision for word $w_i$ at node $j$, $k$ is the length of the path through the tree for word $w_i$, and $node_j$ is the sequence of bits specifying node $j$. As we can see from Equation \ref{eq:hplm_sig}, Morin's hierarchical setup is applied to the neural probabilistic model described in Section \ref{sec:nplm}. Also, the soft-max is replaced by the sigmoid function which plays the same output function role except the normalization occurs over only two outputs.

\begin{figure}
\Tree [.$N_0$ [.$N_1$ {duck}  {rabbit} ] [.$N_2$ {dog} {squirrel} ]  ]
\caption{Simple hierarchical word tree}
\label{fig:htree}
\end{figure}

\paragraph{}
The tree used in the HNPLM is created using expert knowledge from WordNet \cite{MorinBengio2005}. The tree is a balanced tree that guarantees each branch is of length $\log_2(|V|)$. This means that each word can be predicted with $\log_2(|V|)$ normalizations over two outcomes. This is in comparison to the standard NPLM model which has a single normalization over $|V|$ words. For example, using a 10,000 word vocabulary, the HNPLM makes $14$ binary decisions per word. Each decision is over 2 outcomes, for a total of 28 expressions to consider in the normalization. The NPLM, on the other hand, has to consider 10,000 expressions in the normalization. The training and testing times are proportional to the number of expressions normalized over. In this example, we see a 99.72\% reduction in time. The hierarchical set up in general produces an order $O(\frac{|V|}{\log(|V|)})$ speedup.

\paragraph{}
The word itself can also be represented by a binary code corresponding to decisions through the tree. Going left in the tree corresponds to $0$ and going right in the tree corresponds to $1$. For example, in Figure \ref{fig:htree}, the word \emph{dog} corresponds to the code $10$.

\paragraph{}
Morin and Bengio found that the HNPLM produces a less accurate model than the original NPLM but with a $250\times$ speedup. On the Brown corpus, with $|V|$=10,000 and a 900,000 token training set, the NPLM has a test perplexity of 195.3 whereas the HNPLM has a test perplexity of 220.7.

\subsection{Hierarchical Log-Bilinear Model} \label{sec:HLBL}
\paragraph{}
Andriy Mnih and Geoffry Hinton the originators of the log-bilinear model, which is described in Section \ref{sec:lbl},  also explored a faster model: the \emph{hierarchical log-bilinear model} (HLBL). This dissertation is based off of the HLBL work done by Mnih and Hinton. 

\paragraph{}
The HLBL model is described in \cite{MnihHinton2009}. This model uses a tree-based structure to reduce the number of normalizations needed in the LBL model. The speed-up idea is the same as in the HNPLM and also results in a $O(\frac{|V|}{\log|V|})$ speedup. I am choosing to use the HLBL because it directly affects the effective vocabulary component of the training and test times. Since the vocabulary is the largest contributor to the training and test times, reducing the effective vocabulary size should be very good at speeding up the entire model.

\paragraph{}
The formulation is similar to the LBL model. The key difference is that the word probability is now a product of binary probabilities instead of a single soft-max output. Also, there is no longer a separate representation for target words, instead we have representations for nodes in the binary tree. The HLBL formulation is given by:
\begin{align}
&P(w_i | w_{i-(n-1)},\dots, w_{i-1})  \approx \prod_j P(b_j(w_i) | q_j, w_{i-(n-1)},\dots, w_{i-1}) \label{eq:HLBL}
\\ 
&P(b_j(w_i) = 1 | q_j, w_{i-(n-1)},\dots, w_{i-1}) =  \sigma( \hat{r}^T q_{j} +b_{j})
\\ 
&\hat{r} = \sum_{i=1}^{n-1} C_i r_{w_i} \nonumber
\end{align}
where $b_j(w_i)$ is the binary decision for word $w_i$ at node $j$, $q_j$ is the vector representation for node $j$, and $b_{j}$ is a bias term for node $j$. The $r_{w_i}$ term is the vector representation for context word $w_i$, and $C_i$ is the matrix that corresponds to the interaction between the $i$th context word and the target node.  The $\hat{r}$ term can still be thought of as a prediction vector for $w_i$. 

\paragraph{}
Much of the hierarchical formulation, intuition, and speed-up is the same as the HNPLM for the HLBL. The major difference between the two is that the HNPLM has non-linearities whereas the HLBL is linear. This makes the HLBL faster as it does not need to compute non-linearities. Additionally, the prediction vector $\hat{r}$ in the HLBL can be predicted once per word, whereas the full non-linearity must be computed per node for the HNPLM. One other difference is that the HNPLM conditions on all previous binary decisions, whereas the HLBL treats each binary decision independently, which is a simplifying assumption. 

\paragraph{}
The biggest contribution from \cite{MnihHinton2009} is  Mnih and Hinton's exploration of using various types of trees. Mnih and Hinton showed that different types of binary trees influence both model performance and speed. The trees explored included random trees, \emph{BALANCED} trees, \emph{ADAPTIVE} trees, \emph{ADAPTIVE($\epsilon$)}, and overcomplete trees. The creation of such trees is described in~\cite{MnihHinton2009}.
The trees were all created similarly by first training a HLBL using a random tree, and then clustering on the trained average prediction vectors $\bar{\hat{r}}$, which were treated as word representations. The clustering algorithm used was a \emph{mixture of two gaussians}. Both gaussians assigned responsibilities to each word representation. The BALANCED algorithm made splits to maintain a balanced tree, whereas the ADAPTIVE algorithm made splits based on which cluster had a higher responsibility for the word. The ADAPTIVE($\epsilon$) algorithm added the word representation to both clusters if both clusters had responsibilities within $\epsilon$ of 0.5. Clustering was run recursively until a full binary tree was built. 

\paragraph{}
The last trees explored were $2\times$ and $4\times$ overcomplete ADAPTIVE($\epsilon=0.4$) trees. A~$2\times$~overcomplete tree has two ADAPTIVE($\epsilon$) subtrees and a $4\times$ overcomplete tree has four such subtrees. 

\paragraph{}
For the ADAPTIVE($\epsilon$) and overcomplete trees, the same word may have multiple codes. The probability of such words must be the sum over all code instances. The new word probability is given by:
\begin{align}
P(w_i | w_{i-(n-1)},\dots, w_{i-1}) = \sum_{b \in B(w_i)} \prod_j P(b_j(w_i) | q_j, w_{i-(n-1)},\dots, w_{i-1})
\end{align}
where $B(w_i)$ is the set of all codes for word $w_i$.

\paragraph{}
The HLBL models with various trees in decreasing order of performance are: overcomplete, ADAPTIVE($\epsilon$), ADAPTIVE, BALANCED and random. The standard LBL model performance falls somewhere between ADAPTIVE($\epsilon$) trees and overcomplete trees. All of the HLBL models (using various trees) were at least $200\times$ faster than the LBL model. 
The key takeaway is that the type of tree used makes a difference. Trees with multiple branches for difficult words can perform better than the standard LBL model and can also be faster.

\section{Gradient Approximation}
\paragraph{}
Most neural models are trained using gradient descent (or backpropogation which uses gradient descent), so speeding up the gradient calculation can also speed up training of the overall model. Regrettably, gradient approximation techniques only speed up training and not testing. 

\subsection{Importance Sampling}
\paragraph{}
Importance sampling for training neural based models was introduced by Yoshua Bengio and Jean-S\'{e}bastion Sen\'{e}cal in \cite{BengioSenecal2003}.
The idea behind importance sampling is to approximate the negative contribution of the gradient by sampling from an approximate distribution $Q^h_i(w_i)$ for word $w_i$. The $Q^h_i(w_i)$ distribution is given by n-gram models. 

The normal gradient is computed:
\begin{align}
\frac{\partial}{\partial \theta} \log P_{\theta}(w_i| h_i) =& \frac{\partial}{\partial \theta} \hat{r} - \sum_i P_{\theta}(w_i| h_i) \frac{\partial}{\partial \theta} \hat{r_i} 
\\ \frac{\partial}{\partial \theta} \log P_{\theta}(w_i| h_i) =& \frac{\partial}{\partial \theta} \hat{r} - E_{P_{\theta}(w_i| h_i)} \left[ \frac{\partial}{\partial \theta} \hat{r} \right] \nonumber
\end{align}

Importance sampling tries to avoid computing the expectation $E_{P_{\theta}(w_i| h_i)} \left[ \frac{\partial}{\partial \theta} \hat{r} \right]$ as this requires computing $\hat{r}$ for every word in the vocabulary. Instead the gradient is estimated using:

\begin{align}
&\frac{\partial}{\partial \theta} \log P_{\theta}(w_i| h_i) \approx  \frac{\partial}{\partial \theta} \hat{r} - \frac{1}{M} \sum_{j=1}^k m(x_j) \frac{\partial}{\partial \theta} \hat{r}
\\ &m(x) =  \frac{\exp(\hat{r})}{Q^{h_i}(w_i=x)} \nonumber
\\ &M= \sum_{j=1}^k m(x_j) \nonumber
\end{align}
where $k$ is the number of samples taken from $Q^h_i(w_i)$, and $x_j$ is a sample from  $Q^h_i(w_i)$. If $k \ll |V|$ then importance sampling can be much faster than calculating the gradient.

\paragraph{}
Unfortunately, the number of samples $k$ must grow as training progresses to account for the growing variance between the $Q^h_i(w_i)$ distribution and the actual distribution. 
Another way to deal with the growing variance between the $Q^h_i(w_i)$ distribution and the actual distribution is to adapt the $Q^h_i(w_i)$ distribution throughout training as proposed in \cite{BengioSenecal2008}.
\paragraph{}
In \cite{BengioSenecal2003}, it was found that on the Brown corpus, importance sampling produces a $19\times$ speedup over the normal gradient computation.

\subsection{Noise-Contrastive Estimation}
\paragraph{}
Andriy Mnih and Yee Whye Teh use \emph{Noise-Contrastive Estimation} (NCE) for training a LBL model in \cite{MnihTeh2012}. NCE is a more stable sampling method than importance sampling. Additionally, NCE can perform well with fewer samples than importance sampling would need to perform at the same level.
The basic idea in NCE is to discriminate between samples from the actual word distribution and a known noise distribution. In \cite{MnihTeh2012}, a unigram distribution is used for the noise distribution.
The objective function $J$ in NCE is:
\begin{align}
J^h(\theta) = E_{P_d^h} \left[ \log \frac { P_{\theta}^h(w) } {P_{\theta}^h(w) + k P_n(w)} \right] + kE_{P_n} \left[ \log \frac { k P_n(w) } {P_{\theta}^h(w) + k P_n(w)} \right]
\end{align}
where $P_{\theta}^h(w)$ is the actual word distribution and $P_n(w)$ is the known noise distribution. The noise samples are $k$ times more frequent than the word samples.

The gradient can then be calculated:
\begin{align}
\frac{\partial}{\partial \theta} J^{h,w}(\theta) = \frac {  k P_n(w)  } {P_{\theta}^h(w) + k P_n(w)}  \frac{\partial}{\partial \theta}  P_{\theta}^h(w)
- \sum_{i=1}^{k} \left[ \log \frac { P_{\theta}^h(x_i) } {P_{\theta}^h(x_i) + k P_n(x_i)} \frac{\partial}{\partial \theta} \log P_{\theta}^h(x_i) \right]
\end{align}
The NCE gradient formulation's negative contribution is only a sum over $k$ samples, instead of a sum over all words $|V|$ in the vocabulary. If $k \ll |V|$ then there is a significant speedup over the normal gradient calculation.
\paragraph{}
The NCE speedup should be $O \left(\frac{nd + |V|} {nd+ k} \right)$ over the normal gradient computation. Where $n$ is the context-size and $d$ is the word-feature dimensionality.
\paragraph{}
On the Penn Treebank, Mnih and Teh found NCE to be $14\times$ faster than standard \emph{maximum likelihood} (MLE)  training. In \cite{MnihTeh2012}, it is reported that with enough samples, NCE can actually perform slightly better than MLE training on the Penn Treebank. Unfortunately, the better performance is probably due to noise or implicit regularization in NCE training. NCE training should not generally perform better than MLE training.


\section{Vocabulary Truncation}
\paragraph{}
One easy way to reduce the training and testing time, is to simply reduce the number of words. By reducing the vocabulary size, the normalization and expectation over the vocabulary becomes faster. Unfortunately the words that are generally truncated tend to be the least frequent words, which also tend to be the most interesting words for a language model to learn. 
\section{Diagonal Context Matrices}
\paragraph{}
Mnih and Hinton propose diagonal context matrices in \cite{MnihHinton2009}. Neural language models tend to have context matrices $C_i$ that capture the behavior of the context. These context matrices are of size $D \times D$, where $D$ is the dimensionality of the word embeddings. Context matrices are used in computing everything from the objective function to the gradient, so having a smaller context matrix can speedup training and testing of the whole model. One way of having a smaller context matrix is to diagonalize the matrix, so there are effectively only $D$ values in $C_i$. This makes matrix multiplication with $C_i$ faster. Note that this does reduce the representational power of large contexts. Also, the size of $C_i$ is generally much smaller than the size of $|V|$, so reducing $C_i$ will not have a major effect on training times.

\section{Training Order}
\paragraph{}
Curriculum learning proposed in \cite{Bengio2009}, can reduce the number of iterations needed in training. This reduces the overall training time. Training instances are ordered by difficulty, with simple training instances trained before difficult training instances. Intuitively, the model learns the basic structure of the language from simple training cases before tackling more difficult cases. In normal stochastic training, the training instances are fed in random order. This means that it can take longer to learn structure since easy and hard cases are interspersed with each other. Analogously, humans learn faster if we can build up a topic lesson by lesson. If we are forced to learn a hodgepodge of concepts from a topic, it takes us a much longer time to build up a full picture of the topic. Alternatively, in \cite{Mikolov2012}, Mikolov orders his training instances by "usefulness." He starts off with out-of-domain data and ends with the most important in-domain data.


\documentclass[12pt]{ociamthesis}  % default square logo 
%\documentclass[12pt,beltcrest]{ociamthesis} % use old belt crest logo
%\documentclass[12pt,shieldcrest]{ociamthesis} % use older shield crest logo

%load any additional packages
\usepackage{amssymb}
\usepackage[margin=1in]{geometry} 
\usepackage{amsmath,amsthm,amssymb}
\usepackage{listings}% http://ctan.org/pkg/listings
\usepackage{graphicx}
\usepackage [autostyle, english = american]{csquotes}
\usepackage{soul}
\usepackage{qtree}
\usepackage{algorithm2e}


%input macros (i.e. write your own macros file called mymacros.tex 
%and uncomment the next line)
%\include{mymacros}
\MakeOuterQuote{"}
\DeclareMathOperator*{\argmax}{arg\,max}
\DeclareMathOperator*{\argmin}{arg\,min}
\begin{document}

\chapter{Proposed Method}

\section{Proposed model} \label{sec:proposedModel}
\paragraph{}
The hierarchical log bilinear model proposed by Andriy Mnih and Geoffrey Hinton in \cite{MnihHinton2009} is the starting point for my project. Mnih and Hinton make it clear that varying the binary trees used for the hierarchical representation of the vocabulary can make a significant difference in the performance and speed of the model. To be specific, "trees
that are well-supported by the data and are reasonably well-balanced so that the resulting models
generalize well and are fast to train and test" should be used \cite[pg. 5]{MnihHinton2009}. While Mnih and Hinton propose several effective trees to use in the HLBL model, I will explore the effects of other types of binary trees. I intend to test out Huffman encoding trees, Brown Clustering trees, simple overcomplete trees and Recursive ADAPTIVE trees.
\subsection{Motivation}
\paragraph{}
The HLBL model can be made faster and more accurate by using good decision trees. There is a trade-off between the expressiveness of the model's learned node representations and the number of decisions that must be made per word. In the ideal situation, the learned node representations would exactly capture their constituent words' semantic/lexical properties and also the correct decision path through the tree. In absence of the ideal, we would like to make the decision process easier to learn so the representations can capture the semantic/lexical properties better. Minh and Hinton do a good job creating trees that alleviate this burden, but I would like to explore even simpler trees in the hope that they can also capture useful decisions. Hopefully, trees that are simple to create and understand will also perform well in the HLBL model. 

\subsection{Model Definition}
\paragraph{}
The model used will be the HLBL model described in section \ref{sec:HLBL}. The HLBL treats each word probability as the product of the probabilities of binary decisions through a tree.

\subsubsection{Maximum Likelihood Objective Function}
\paragraph{}
The objective function ($J$) , also known as the \emph{likelihood}, for the HLBL model is the product of all word probabilities in the training set. The objective function can be thought of as the joint probability of all words in the training set. We would like to maximize the objective function. 
\begin{align}
&J(\theta;w_i,\dots, w_m) = \prod_{i=1}^{m} P(w_i | w_{i-(n-1)},\dots, w_{i-1})
\\
&P(w_i | w_{i-(n-1)},\dots, w_{i-1})  \approx \prod_j P(b_j(w_i) | q_j, w_{i-(n-1)},\dots, w_{i-1})
\\ 
&P(b_j(w_i) = 1 | q_j, w_{i-(n-1)},\dots, w_{i-1}) =  \sigma( \hat{r}^T q_{j} +b_{j}) \nonumber
\\ 
&P(b_j(w_i) = 0 | q_j, w_{i-(n-1)},\dots, w_{i-1}) =  1- \sigma( \hat{r}^T q_{j} +b_{j}) \nonumber
\\ 
&\hat{r} = \sum_{i=1}^{n-1} C_i r_{w_i} \nonumber
\\  
&\sigma(x) = \frac{1}{1+e^{-x}} \label{eq:sigmoid} \nonumber
\end{align}

Where $\theta$ is shorthand for all of the parameters of the model: $Q,R,B$,and $C_i$ for all context positions. $m$ is the size of the training set.  $b_j(w_i)$ is the binary decision for word $w_i$ at node $j$. $q_j$ is the vector representation for node $j$, and $b_{j}$ is a bias term for node $j$. $r_{w_i}$ is the vector representation for context word $w_i$, and $C_i$ is the matrix that corresponds to the interaction between the $i$th context word and the target node.  $\hat{r}$ can be thought of as a prediction vector for $w_i$. Note that there is a simplifying assumption that binary decision predictions are independent of each other.
\paragraph{}
We would like to choose parameters ($\hat{\theta}$), such that $\hat{\theta} = \argmax_\theta J(\theta;w_i,\dots, w_m)$. $\hat{\theta}$ is known as the \emph{maximum likelihood estimator} of $\theta$ \cite{Elkan2013}.  In our actual usage of the objective function we will modify it slightly. We can take the negative natural logarithm of the objective function to get the negative log likelihood. Charles Elkan states that "[b]ecause logarithm is a monotonic strictly increasing function, maximizing the log likelihood is precisely equivalent to maximizing the likelihood, and also to minifmizing the negative log likelihood" \cite[pg. 3]{Elkan2013}. So instead of maximizing the likelihood, we now minimize the negative log likelihood such that  $\hat{\theta} = \argmin_\theta -ln(J(\theta;w_i,\dots, w_m))$.
Using some logarithm rules we see that:
\begin{align}
&-ln(J(\theta;w_i,\dots, w_m)) = \sum_{i=1}^{m} -ln(P(w_i | w_{i-(n-1)},\dots, w_{i-1}))
\\
&-ln(P(w_i | w_{i-(n-1)},\dots, w_{i-1})) \approx \sum_j - ln(P(b_j(w_i) | q_j, w_{i-(n-1)},\dots, w_{i-1})) 
\\
&- ln(P(b_j(w_i) = 1 | q_j, w_{i-(n-1)},\dots, w_{i-1})) = ln (1 + e^{-\hat{r}^T q_{j} -b_{j}}) \nonumber
\\
&- ln(P(b_j(w_i) = 0 | q_j, w_{i-(n-1)},\dots, w_{i-1}) = ln (1 + e^{\hat{r}^T q_{j} +b_{j}}) \nonumber
\end{align}

\paragraph{}
We use the logarithm of the objective function because it changes the products into sums which are easier to calculate and are less likely to overflow or underflow on a computer. Additionally, using logarithms can make the gradients easier to derive as natural log is very easy to differentiate.


\subsubsection{Probability Mass Function}
\paragraph{}
The word probability distribution is given by a \emph{probability mass function}(PMF) that sums up to one. We can see that the PMF must sum to one if we look at the sigmoid output function on each binary decision. The sigmoid function given by equation \ref{eq:sigmoid} is always between $0$ and $1$. We can see this in figure \ref{fig:sigmoid}. This means that we can treat each binary decision output as a probability. If we sum over both possibilities for a binary decision, the probability will be 1. We can therefore think of each decision as dividing up the probability mass, but never losing any mass. 
In figure \ref{fig:btree}, the four word probabilities $(\frac{1}{3},\frac{1}{3}, \frac{1}{12}, \frac{1}{4})$ would each be the product of decision probabilities on the path to their respective word. These word probabilities sum up to one.

\begin{figure}
\centering
\includegraphics[height=200px]{./images/sigmoid.png}
\caption{Sigmoid function}
\label{fig:sigmoid}
\end{figure}

\begin{figure}
\Tree [.$dp=1$ [.$dp=\frac{2}{3}$ [.{$dp=\frac{1}{2}$} {$wp = \frac{1}{3}$} ]  [.{$dp=\frac{1}{2}$} {$wp = \frac{1}{3}$} ]] [.$dp=\frac{1}{3}$ [.{$dp=\frac{1}{4}$} {$wp = \frac{1}{12}$} ] [.{$dp=\frac{3}{4}$} {$wp = \frac{1}{4}$} ] ]  ]
\caption{Binary decision probabilities in a tree. dp is a decision probability and wp is a word probability}
\label{fig:btree}
\end{figure}

\subsubsection{Regularization}
\paragraph{}
Standard maximum likelihood estimation solutions tend to overfit. This over fitting comes from the fact that parameter values are unbounded, so in the limit, all parameter values will go to infinity when maximizing the overall objective function. To stop parameters from going to infinity, \emph{regularization} is added. Regularization simply adds a penalty to large parameter values. Many types of penalties exist, but I will be using the squared $L_2$ norm ($||x||^2_2$) as a penalizer. The maximum likelihood estimator then becomes:
\begin{align}
\hat{\theta} = \argmin_\theta -ln(J(\theta;w_i,\dots, w_m)) +  \mu ||\theta||^2_2
\\
||\theta||^2_2 = \sum_{j=1}^d \theta_j^2 \nonumber
\end{align}
Where $\mu$ is the regularization parameter that controls the tradeoff between maximizing J and minimizing the parameter values. $d$ is the length of the parameter vector $\theta_j$ \cite{Elkan2013}.

\subsubsection{Similarity to Logistic Regression}
\paragraph{}
The HLBL model is actually very similar to logistic regression. We can think of HLBL as logistic regression with knowledge of context words. 
The standard logistic regression model looks like:
\begin{align}
p(y =1 | x; \alpha, \beta) = \sigma \left( \sum_{j=1}^d \beta_j x_j + \alpha \right)
\end{align}
If we reformulate the logistic regression model to look like the HLBL:
\begin{align}
p( b_j(w_j) = 1 | w_i; b_j, q_j) = \sigma \left( \sum_{j=1}^d q_j w_j + b_j \right)
\end{align}
We can then compare it against the HLBL model:
\begin{align}
P(b_j(w_i) = 1 | q_j, w_{i-(n-1)},\dots, w_{i-1}) =  \sigma( (\sum_i C_i r_{w_i})^T q_{j} +b_{j})
\end{align}
We see that most of the two models are the same, the main difference being the addition of the $C_i$ matrices.  The $C_i$ matrices hold parameters that deal with the interactions between context words and nodes in the tree. $q_j$ also plays a different role. In the logistic regression model, $q_j$ is a weight for word $w_j$, whereas in the HLBL model, $q_j$ is a vector representation for a node in the tree. Still, the models are very similar and we can think of HLBL as logistic regression for nodes in the tree, weighted by context prediction vectors.


\subsection{Model training}
\paragraph{}
To train the HLBL model, we need a way to minimize the MLE parameters $\hat{\theta}$. One way to minimize the parameters is through an iterative scheme that is guaranteed to move closer to the (local) minimum on each step called gradient descent. The HLBL objective function is smooth and so is easily differentiable. However, it is not convex so it may have more than a one minimum. 
\subsubsection{Gradient Descent}
\paragraph{}
\emph{Gradient descent} is a first-order optimization algorithm. The basic idea behind gradient descent is to move in the direction of the negative gradient on each time step until the gradient becomes zero. When the gradient is zero, we have hit a local minimum. Since the MLE parameters minimize $-J(\theta)$, we can apply gradient descent to the parameters $\theta$. Of course, even if we are constantly moving in the direction of the gradient, we may be moving towards it in too small steps or too large steps. This is why we include a learning rate $\gamma$, to control the step size at which we move toward the minimum. The learning rate $\gamma$ is typically fixed, and found by trial and error \cite{Elkan2013}.
A single step of gradient descent is given by:
\begin{align}
\theta := \theta - \gamma \sum_{i=1}^{m} \nabla_\theta \left( -ln(J(\theta;w_i,\dots, w_m)) +  \mu ||\theta||^2_2 \right) \label{eg:gradientDescent} 
\end{align}
Where $m$ is the total number of training examples.

\subsubsection{Gradients}
\paragraph{}
We can break up equation \ref{eg:gradientDescent} into its constituent gradients for parameters $Q,R,B$ and $C_i$ for all context indices. The gradient derivation can be found in \hl{Appendix A}.

Here are the gradients after every decision $b_j(w_i)$:
\begin{align}
\nabla_{b_j} \left( -ln(J(\theta;w_i,\dots, w_m)) +  \mu ||\theta||^2_2 \right)  = &  \left( P \left(b_j(w_i) = 1 | h \right) - b_j(w_i) \right) +2\mu b_j  \label{eq:gradients}
\\
\nabla_{q_j} \left( -ln(J(\theta;w_i,\dots, w_m)) +  \mu ||\theta||^2_2 \right)  = &\left( P \left(b_j(w_i) = 1 | h \right) - b_j(w_i) \right)\hat{r} +2\mu q_j \nonumber
\\
\nabla_{r_{w_i}} \left( -ln(J(\theta;w_i,\dots, w_m)) +  \mu ||\theta||^2_2 \right)  = & \left( P \left(b_j(w_i) = 1 | h \right) - b_j(w_i)\right) C_i^T q_j   +2\mu r_{w_i} \nonumber
\\
\nabla_{C_i} \left( -ln(J(\theta;w_i,\dots, w_m)) +  \mu ||\theta||^2_2 \right)  = &\left( P \left(b_j(w_i) = 1 | h \right) - b_j(w_i) \right) r_{w_i}q_j^T  +2\mu C_i \nonumber
\\
P \left(b_j(w_i) = 1 | h \right)= & P \left(b_j(w_i) = 1 | q_j, w_{i-(n-1)},\dots, w_{i-1} \right) \nonumber
\end{align}

$q_j$ and $b_j$ should be updated per decision but $r_{w_i}$ and $C_i$ should only be updated per word. Here are the word gradients:
\begin{align}
& \nabla_{r_{w_i}} \left( -ln(J(\theta;w_i,\dots, w_m)) +  \mu ||\theta||^2_2 \right)  =  C_i^T \sum_j \left( P \left(b_j(w_i) = 1 | h \right)  - b_j(w_i) \right)q_j +2\mu r_{w_i} \nonumber
\\
& \nabla_{C_i} \left( -ln(J(\theta;w_i,\dots, w_m)) +  \mu ||\theta||^2_2 \right)  =  r_{w_i} \sum_j \left( P \left(b_j(w_i) = 1 | h \right) - b_j(w_i) \right) q_j^T  +2\mu C_i
\end{align}
Gradient descent updates the parameters once for every full sweep though all of the training examples. A single iteration of gradient descent takes order $O(M\times D)$, where $M$ is the number of training examples and $D$ is the dimensionality of the feature vector \cite{Elkan2013}. 

\subsubsection{Stochastic Gradient Descent}
\paragraph{}
The \emph{stochastic gradient descent} algorithm is almost the same as gradient descent except that all of the parameters are updated after each training example. Also, the training examples are trained in random order. Intuitively, stochastic gradient descent approximates the gradient over all examples, by using a single random example. Stochastic gradient decent can converge to good parameter values in much fewer training iterations than gradient descent. A single iteration of stochastic gradient descent takes order $O(M\times F \times D)$ where $M$ is the number of training examples, $D$ is the dimensionality of the feature vector and $F$ is the average number of non-zero features per example \cite{Elkan2013}. 
A stochastic gradient descent update looks like:
\begin{align}
\theta := \theta - \gamma \nabla_\theta \left( -ln(J(\theta;w_i,\dots, w_m)) +  \mu ||\theta||^2_2 \right)
\end{align}
\subsubsection{Mini-batches}
\paragraph{}
A compromise between gradient descent and stochastic gradient descent is to use mini-batches. Instead of updating parameters after each example like stochastic gradient descent, or after all examples like gradient descent, one updates the parameters after a mini-batch of examples. A \emph{mini-batch} is a small selection of examples. The size of a mini-batch is typically a small percentage of the total parameters: on the order of a few hundred or thousand depending on how many total parameters there are.


A single step of stochastic gradient descent using mini-batches looks like:
\begin{align}
\theta := \theta - \gamma \sum_{i=1}^{K} \nabla_\theta \left( -ln(J(\theta;w_i,\dots, w_m)) +  \mu ||\theta||^2_2 \right)
\end{align}
Where $k$ is the mini-batch size.
\paragraph{}
A single iteration of stochastic gradient descent with mini-batches takes order $O(M\times F \times D\times K)$ where $K$ is the mini-batch size, $M$ is the number of training examples, $D$ is the dimensionality of the feature vector and $F$ is the average number of non-zero features per example. In this dissertation, I will be training my models using stochastic gradient descent with mini-batches.

\subsection{Model testing}
\paragraph{}
As described in section \ref{sec:perplexity}, I will be comparing my models using their test perplexity.
\subsubsection{Perplexity}
We can fill in the equation for perplexity given by equation \ref{eq:perplexity} now that we know our model definition.
\begin{align}
&PP(w_i | h_i)=e^{- \frac{1}{K} \sum_{i=1}^K ln( P(w_i | h_i) ) }
\\
&ln(P(w_i | h_i )) = \sum_j ln(P(b_j(w_i) | q_j, h_i)) \nonumber
\\
&ln(P(b_j(w_i) | q_j, h_i)) =  
\begin{cases}
  -ln (1 + e^{-\hat{r}^T q_{j} -b_{j}}) & \text{if } b_j(w_i)  = 1 \\
  -ln (1 + e^{\hat{r}^T q_{j} +b_{j}})     & \text{if } b_j(w_i) = 0
  \end{cases} \nonumber
\\
 & h_i = ( w_{i-(n-1)},\dots, w_{i-1} ) \nonumber
\end{align}
Where $K$ is the number of testing examples. 


\section{Tree creation} \label{sec:treeCreation}
\paragraph{}
The binary tree used in the HLBL model can make a significant difference to the model performance and speed. In this dissertation, I will explore four different types of trees.
\subsection{Huffman Trees}
\paragraph{}
\emph{Huffman coding} is an optimal prefix code created using a bottom-up agglomerative approach. Huffman coding was introduced by David Huffman in \cite{Huffman1952}. Huffman codes minimize the average number of bits needed to convey a message. In my usage, the Huffman code minimizes the average number of bits to convey a word in the vocabulary. In Huffman coding, each word has a single unique code.
\paragraph{}
I want to explore using Huffman Trees in the HLBL model because Huffman trees are guaranteed to be the most optimal code in terms of minimum average length. This will speed up the model, because on average, the model will make as few decisions as possible. Additionally, I am assuming that as the number of decisions goes up, the overall accuracy of those decisions goes down. With a Huffman code, frequent words will have short codes, therefore it will be more likely that frequent words will be correctly predicted. This should positively affect the model performance since a large percentage of words should have high probabilities of being predicted correctly. On the other hand, the most interesting words for a language model are actually the infrequent words.  Since Huffman trees are tied to word frequencies, the decision nodes will not have any correlation to word semantics. This makes the tree harder for humans to understand and perhaps harder for the model to learn the node representations.  Huffman trees suffer from the same problem as frequency-binned classes described in section \ref{sec:frequencybinning}.
\paragraph{}
As Huffman codes are variable length codes, it is important that Huffman codes are also prefix codes. The prefix property states that no code appears as the prefix for another code. This makes decoding possible without knowing code lengths and also makes representing the codes as a binary tree possible. 
\paragraph{}
To create a Huffman code, a probability distribution over all words must be known. Given that we have an exact unigram distribution of the training data, Huffman coding is a very appealing option. To create a Huffman code, first, each word is assigned to its own cluster.  The clusters are then arranged in order of least probable to most probable. Next, the two least probable clusters are merged together recursively until there is only one cluster left. The probability of a cluster is simply the sum its constituent word probabilities. In this way, shorter codes will always represent more frequent words, and longer codes will represent more infrequent words. Huffman codes create a single unique code per word.
\paragraph{}
The Huffman tree is created while computing the Huffman code. Each word is a leaf in the tree and each merge can be seen as a decision node in the tree. A decision node's children are the two nodes that were merged together. The Huffman code for a word is the list of binary decisions taken from the root node to the word node.

\subsubsection{Huffman Tree Pseudo-code}

\begin{algorithm}[H]
\SetAlgoLined
 \KwData{$w_1,\dots,w_{|V|}$, $P(w_1),\dots,P(w_{|V|}$)}
 \KwResult{ a single node $n_1$}
Put each word $w_1,\dots,w_{|V|}$ into its own node: $n_1\dots,n_{|V|}$
\\ Each node $n_i$ has weight $P(w_i)$
\\Enqueue all nodes $n_1\dots,n_m$ in $Queue1$.
\\Sort $Queue1$ in increasing order so that the smallest weight is at the head
\\ \While{ size($Queue1$) + size($Queue2$) $>$ 1} {
	$n_1$ = dequeued lowest weight node in $Queue1 \cup Queue2$
	\\$n_2$ = dequeued second lowest weight node in $Queue1 \cup Queue2$
	\\$n_3$ = new node with n1 and n2 as children,  $n_3$'s weight is $weight(n_1)+weight(n_2)$
	\\enqueue $n_3$ in rear of $Queue2$
}
$n_1$ = dequeued node from $Queue2$
\\ \Return $n_1$
\end{algorithm}


\subsubsection{Creation Complexity}
\paragraph{}
Huffman encoding takes on the order of $O\left(|V| log |V| \right)$ calculations. The initial sorting takes $O\left(|V| log |V|\right)$ calculations and the queue processing takes $O(|V|)$ calculations.

\subsection{Brown Clustering Trees}
\paragraph{}
\emph{Brown clustering} clusters words into classes and is a bottom-up agglomerative technique that starts off with each word in its own class, and then successively merges the classes based on some criteria. For my purposes, I will merge classes until there is only a single class left.  Brown clustering is introduced by Peter Brown, et al. in \cite{Brown1992}.
Brown clusters try to maximize the average mutual information of adjacent classes.
\paragraph{}
I am interested in using Brown clustering because by maximizing the average mutual information of adjacent classes, Brown clusters tend to cluster words that co-occur together. Words that co-occur are often semantically similar words. Hopefully this means decisions in the Brown clustering tree will take into account semantic differences or at least co-occurence information. This deals with the major shortfall of Huffman trees. Mnih and Hinton do consider Brown clustering trees in \cite{MnihHinton2009}, but dismiss it due to the fact that each word only belongs to a single class. 
\paragraph{}
The average mutual information of adjacent classes (I) is defined as:
\begin{align}
& I(c_1,c_2) =\sum_{c_1,c_2} P(c1,c2)  log \left( \frac{P(c_2|c_1)}{P(c_2)} \right)
\\
&P(c) = \frac{count(c)}{T} \nonumber
\\
&P(c_2|c_1) = \frac{count(c1,c2)}{\sum_c count(c_1,c)} \nonumber
\end{align}
Where $count(c)$ is the number of training words in class $c$ and $T$ is the size of the training set.
\paragraph{}
Brown describes a greedy algorithm that tries to maximize $I$ of all classes, "we assign each word to a distinct class and compute the average mutual information 
between adjacent classes. We then merge that pair of classes for which the loss in 
average mutual information is least. After V - C of these merges, C classes remain. 
Often, we find that for classes obtained in this way the average mutual information 
can be made larger by moving some words from one class to another. Therefore, after 
having derived a set of classes from successive merges, we cycle through the vocabu- 
lary moving each word to the class for which the resulting partition has the greatest 
average mutual information. Eventually no potential reassignment of a word leads to 
a partition with greater average mutual information. At this point, we stop" \cite[pg. 472]{Brown1992}.  
Note that Brown clustering results in each word belonging to a single cluster. Also, Brown clusters with a mutual information criteria tend to put together words that are semantically similar.
\paragraph{}
This algorithm can be improved upon, as Percy Liang demonstrates in \cite{Liang2005}. The new algorithm is very similar except that only the top $m$ most frequent words are put into their own clusters to begin with. The next words are then merged into one of the $m$ clusters based on some criteria. Once all the words are in one of the $m$ clusters, the $m$ clusters are merged based on the same criteria.
The criteria used by Liang is:
\begin{align}
Quality(C) = \sum_{c,c'} I(c,c') - H
\\
H= -\sum_w P(w)logP(w) \nonumber
\end{align}
Where H is the entropy and $\sum_{c,c'} I(c,c')$ is the average mutual information over all classes.
\paragraph{}
The binary tree for Brown clustering is created the same way as it is for Huffman coding. Each word is represented using a leaf node. A new decision node is created upon each merge. The code for every word is then the decisions taken from the root to the word.

\subsubsection{Brown-Liang Clustering Pseudo-code}
The pseudo-code is outlined in \cite{Collins2011}.
\begin{algorithm}
\SetAlgoLined
 \KwData{$w_1,\dots,w_{|V|}$}
 \KwResult{ a single cluster $c_1$}
Take the top $m$ most frequent words and put each into its own cluster: $c_1\dots,c_m$

\For{ $i=(m+1),\dots, |V|$ } {
	$c_{m+1}$ = $i$th most frequent word
	\\
	merge two clusters $c_a$ and $c_b$ such that Quality(C) is maximized
	
}
\For{ $i=1,\dots, m-1 $ } {
merge two clusters $c_a$ and $c_b$ such that Quality(C) is maximized
}
\Return $c_1$
\end{algorithm}

\subsubsection{Creation Complexity}
\paragraph{}
Calculation of the original Brown clusters takes $O(|V|^5)$ calculations and Brown-Liang clusters requires $O(|V| m^2+n|)$ calculations. $m$ is the number of initial clusters and $n$ is the corpus length.  

\subsection{Overcomplete Trees}
\paragraph{}
An idea introduced by Mnih and Hinton in \cite{MnihHinton2009} is the use of overcomplete trees. An overcomplete tree is simply the concatenation of $F$ full vocabulary trees. The trees are connected by a new node at the root; the vocabulary trees become subtrees of the new root. 
\paragraph{}
Overcomplete trees are a good idea because they allow the HLBL model to take advantage of multiple representations for each word. It may be that certain words should have multiple representations such as polysemous words, or it could be that certain trees are better for representing a word. By using overcomplete trees, the HLBL model is allowed the freedom to choose between the varying code representations while only adding $log_2(F)$ more levels to the tree.
\paragraph{}
In \cite{MnihHinton2009}, only overcomplete trees with $ADAPTIVE(\epsilon)$ subtrees are used. I would like to see how using other types of subtrees can affect the performance of overcomplete trees. I will look at $2\times$ overcomplete trees, where there are two subtrees. While many combinations are possible, I will focus on looking at random and random overcomplete trees, random and Huffman overcomplete trees, Huffman and Brown cluster overcomplete trees, and also Huffman and reduced Huffman overcomplete trees. 
\paragraph{}
My idea for a reduced Huffman tree is to create a tree of less frequent words. Since the Huffman tree already orders words by frequency, if I simply cut the tree at some depth $d$, all nodes below depth $d$ should contain words that are infrequent, with larger $d$ being more infrequent words. After cutting the tree, the nodes can be reformed into a tree using the normal Huffman code algorithm. Using reduced Huffman trees in overcomplete trees will give less frequent words more code representations, and hopefully will result in better accuracy for less frequent words.

\subsubsection{Creation Complexity}
\paragraph{}
The creation complexity of overcomplete trees is simply the summation of the creation complexities of each subtree.

\subsection{Recursive ADAPTIVE Trees}
\paragraph{}
Another tree introduced in \cite{MnihHinton2009} is the ADAPTIVE tree. The ADAPTIVE tree is also mentioned in section \ref{sec:HLBL}. The ADAPTIVE tree is created top-down by clustering on learned word representations. The clustering method is a mixture of two Gaussians. The clusters are created with respect to the most responsible Gaussian, which generally creates a non-balanced tree. The real power behind this method comes from the learned word representations which are better than unigram or co-occurance statistics that inform Huffman and Brown cluster trees. The ADAPTIVE tree is in tune with the model, and creates decision nodes that are easy to navigate for the model. 
\paragraph{}
Mnih and Hinton use learned word representations from the HLBL model with a random tree. I think starting off with a better tree such as a Huffman tree or a Brown Cluster tree for the initial HLBL model can improve the ADAPTIVE tree performance. Additionally, I think recursively training the ADAPTIVE tree, on the results from the HLBL model with an ADAPTIVE tree could be very fruitful. Essentially by recursively adapting the tree, the tree can better capture the learned word representations, which results in better learned word representations. 
\subsubsection{Recursive ADAPTIVE Pseudo-code}

\begin{algorithm}
\SetAlgoLined
RecursiveADAPTIVE: \\
 \KwData{ HLBL model M, tree T}
 \KwResult{ Recursive ADAPTIVE tree T}

\While{ perplexity(M) keeps going down } {
 	Get the average prediction vector $\bar{\hat{r}}_{w_i}$ for each word from M
 	\\R = ($\bar{\hat{r}}_{w_1}$,\dots,$\bar{\hat{r}}_{w_m}$)
	\\T = ADAPTIVE(R,empty tree);
	\\M = train HLBL model using T
}
\Return T
\end{algorithm}

\begin{algorithm}
\SetAlgoLined
ADAPTIVE:
 \KwData{ set of word vectors R}
 \KwResult{ tree T}
\If {$size(R) <= 2$}{
	create leaf nodes for words in R and add to T
}
\Else{
	responsibilities = GaussianMixtureModel(R)
	\\ \If{ responsibility of $Gaussian_1$ for $\bar{\hat{r}}_{w_i}$ $>$  responsibility of $Gaussian_2$ for $\bar{\hat{r}}_{w_i}$} {
		add $\bar{\hat{r}}_{w_i}$ to $R_1$
	}
	\Else{
		add $\bar{\hat{r}}_{w_i}$ to $R_2$
	}
	\\ create a decision node $n_1$ and add to T
	\\ $n_1$.leftChild = ADAPTIVE($R_1$)
	\\ $n_1$.rightChild = ADAPTIVE($R_2$)
}
\Return T
\end{algorithm}

\begin{algorithm} \label{code:GMM}
\SetAlgoLined
GaussianMixtureModel:
\\
 \KwData{ set of word vectors R}
 \KwResult{ responsibilities $\gamma$}
Initialize means $\mu_k$, covariances $\sigma_k$ and mixing coefficients $\pi_k$
\\Evaluate the initial log likelihood
\\
\While{parameters have not converged and log likelihood has not converged}{
	E step:
	\\ $\gamma(z_{nk}) = \frac{ \pi_k \mathcal{N}(R_n|\mu_k,\sigma_k) }{\sum_{j=1}^K \pi_j \mathcal{N}( R_n| \mu_j, \sigma_j) }$
	\\ M step:
	\\ $N_k = \sum_{n=1}^N \gamma(z_{nk})$
	\\ $ \mu_k^{new} = \frac{1}{N_k} \sum_{n=1}^N \gamma(z_{nk}) R_n$
	\\ $\sigma_k^{new} = \frac{1}{N_k}\sum_{n=1}^N \gamma(z_{nk}) (R_n - \mu_k^{new}) (R_n - \mu_k^{new})^T $
	\\ $ \pi_k^{new} = \frac{N_k}{N}$
	\\ Evaluate the log likelihood:
	\\ $ln( Pr(R|\mu,\sigma,\pi)) = \sum_{n=1}^N ln \left( \sum_{k=1}^K \pi_k \mathcal{N}(R_n| \mu_k, \sigma_k) \right)$
}
\Return $\gamma$
\\
Gaussian Mixture Model pseudo-code from \cite{Bishop2006}
\end{algorithm} 

\subsubsection{Creation Complexity}
\paragraph{}
The creation complexity of a Recursive ADAPTIVE tree is $O ( |V| \times log|V| \times G + (R-1)\times M + (R-1)\times T)$ where $R$ is the number of iterations it takes the Recursive ADAPTIVE algorithm to converge, $G$ is the number of iterations it takes the Gaussian Mixture Model to converge, $M$ is the maximum training complexity of one of the previous recursive HLBL models, and $T$ is the maximum creation complexity of one of the trees used by the recursive HLBL models. 

%next line adds the Bibliography to the contents page
\addcontentsline{toc}{chapter}{References}
%uncomment next line to change bibliography name to references
\renewcommand{\bibname}{References}
\bibliography{refs}        %use a bibtex bibliography file refs.bib
\bibliographystyle{plain}  %use the plain bibliography style

\end{document}


\chapter{Account of Work}
\paragraph{}
In this chapter I will describe the work carried out in this project. I will first describe engineering tasks undertaken in the course of the project. Secondly, I will define the experiments and state the experimental results. Lastly, I will analyze the results and their implications.

\section{Engineering Details}
\paragraph{}
My implementation of the HLBL model is an extension of the C++ LBL implementation by Phil Blunsom, et al. Additionally, I use tree.hh: an STL-like C++ tree library written by Kasper Peeters, and the Boost \cite{BoostSite} and Eigen \cite{eigenweb} libraries. The full implementation can be found online at \url{https://github.com/jeannewang/oxlm}. Much of the code can be found in Appendix C.
\paragraph{}
In this section, I will describe in more detail the engineering tasks necessary to build the HLBL model, described in section \ref{sec:proposedModel}. These tasks include initializing the model, building binary trees, checking the finite gradient and also caching key parameters.

\subsection{Model Initialization}
\paragraph{}
All of the model parameters ($Q,R,$ and $C_i$ for all contexts) except for the B vector are initialized using Gaussian distributions with mean $0$, and variance $0.1$.
\subsubsection{B Initialization}
\paragraph{}
Intelligent initialization of the bias matrix can significantly improve perplexity measures. The $B$ vector contains a bias for each node in the binary tree.  The $B$ vector should be initialized with the unigram distribution of the training data. As a bias $b_j$ does not directly correlate to an individual word, $b_j$ should be initialized to the sum of the unigram probability of all words in the subtree under node $j$.  Compared to initialization with a Gaussian distribution, there is a $50\times$ improvement in initial perplexity measures. If the model's initial perplexity is lower, then fewer training iterations are generally needed. I also found that a poor initial perplexity often negatively affects the perplexity of the fully trained model.

\subsection{Building Trees}
\paragraph{}
I have described the algorithms to build specific trees in section \ref{sec:treeCreation}, but I will describe here some additional tips for building binary trees. I use simple integer trees for my binary trees. My trees contain indices into the $Q$ matrix for the internal node values, and indices into the $R$ matrix for the leaf values. The indices into the $R$ matrix are also the word indices. To tell if I am looking at a word, or a node, I simply check if the node is a leaf node or not.  I found building binary trees for vocabularies to be easier going bottom-up than going top-down. This is because one can group together "similar" words much more simply by going from the words to the nodes, than the other way around. I also found that the easiest way for me to keep track of indices to the $Q$ matrix is by only storing the indices in the tree. Since the $Q$ matrix is initialized with a Gaussian, it makes no difference which $Q$ node is assigned to which $Q$ index so long as it is consistent. I just assign $Q$ indices sequentially breadth-first though the tree. 
\paragraph{}
When reading in trees from binary codes, it is necessary to build the tree top-down. To build the tree, I read in every code, and for each digit in the code, add in a node if necessary. If a new code has a branch that does not exist in the tree, I simply create it, and continue parsing the code.

\subsubsection{Mixture of Gaussians for ADAPTIVE trees}
\paragraph{}
A \emph{Gaussian Mixture Model} (GMM) is used to cluster word representations in the ADAPTIVE and Recursive ADAPTIVE trees described in section \ref{code:GMM}. To initialize the parameters for the GMM, I randomly assign the word representations into two groups, and from there calculate the group means and covariances to initialize $\mu_k$ and $\sigma_k$. The mixing coefficients $\pi_k$ are initialized to the uniform distribution. Also, when calculating the Gaussian, I found it much faster to use diagonalized versions of the covariance matrices. While this does lose some representational power of the covariance matrices, I find the speedup to be worth it. I run each GMM until convergence, or for 10 epochs, whichever is less. If any of the parameters become ill-defined, I simply back off to the previous epoch's values. 

\subsection{Finite Gradient Check}
\paragraph{}
A very easy place to mess up the HLBL implementation is in the gradient computation. One easy way to check if the gradients are being calculated correctly is to use a finite gradient check. The finite gradient check uses the mathematical definition of the derivative. I am going to use the central difference definition, though the forward and backward differences can also be used. The derivative is given by:
\begin{align}
\frac{d}{d\theta} J(\theta) = \lim_{\epsilon \to 0} \frac{ J({\theta + \epsilon} )- J({\theta - \epsilon}) } {2 \epsilon}
\end{align}
We see that the derivative is given in the limit of $\epsilon \to 0$, therefore if we set $\epsilon$ to a small number ($\approx10^4$), we can approximate the derivative of the objective function. The finite gradient check can then be compared with the gradients calculated from equations \ref{eq:gradients} on the training data. The finite gradient check is very computationally slow, and so should only be used as a check and not as the final gradient computation.

\subsection{Caching}
\paragraph{}
Certain variables should be cached as they are used multiples times while calculating portions of the objective function, the gradients or the perplexity. It is important to cache these values as the caching can significantly reduce the training time by avoiding recomputing computationally-expensive values. The values that should be cached include: $\hat{r}$, and $C_i^T q_j$. $\hat{r}$ is particularly important to cache as it is used in the calculation of the objective function, all of the gradients, and the perplexity. Note that a separate $\hat{r}$ should be cached for training versus testing data.

\section{Experiments} \label{sec:experiments}
\paragraph{}
I have conducted experiments to show the performance and speed of the HLBL model with various trees. For all of the models, I tried to use similar parameter values for comparison's sake. I used a word and node representation size of $100$ dimensions, a $5$-word context (including the target word), a mini batch size of $10,000$ and a learning step-size of $0.01$. I also used a small regularization constant while training all of the models. Most of the models could actually be learned faster if the parameters were tuned to the individual model. The models were all trained on the same machine using a single thread. The machine runs at 3.47GHz and has 60GB of RAM.

\subsection{Datasets}
\paragraph{}
For most of my experiments, I use the Penn Treebank dataset. Specifically, I use the Wall Street Journal (WSJ) sections 2-21 for my training set and section 22 for my testing set. There are 950028 training tokens and 40117 testing tokens. The training set has been pre-processed to replace any word that occurs less than 6 times with "\_UNK\_". This reduces the vocabulary to 10531 words. The testing set has also been preprocessed to replace any word not seen in the training set with "\_UNK\_".
\subsubsection{Large Vocabulary}
\paragraph{}
Real languages have much larger vocabularies than the WSJ dataset. The Oxford English Dictionary currently has 171,476 full word entries for English \cite{OED}. For a morphologically rich language, the vocabulary size can easily exceed that of English.
\paragraph{}
The HLBL model really shines in terms of speed with very large vocabulary sizes. The speedup using the HLBL model is order $O(\frac{|V|}{log|V|})$. To highlight this, I also run some experiments where I use a subset of the British National Corpus (BNC). I use the first 120,000 lines of the BNC, where 100,000 lines are randomly assigned to the training set and the other 20,000 lines are assigned to the test set. There are 2141639 training tokens and 427948 testing tokens. The training set has a vocabulary size of 84439 (including "\_UNK\_").The test set is preprocessed to replace any word not seen in the training set with "\_UNK\_". We can see on the 85K word vocabulary where non-HLBL models slow down immensely.


%WSJ sections 2-21, with words that occur only once preprocessed out, as a training set. This set has a vocabulary size of 23768. 

%\subsection{Varying Trees}
%I use random Trees, Huffman trees, Brown Cluster trees, ADAPTIVE trees, and Recursive Adaptive Trees with the HLBL model.

\section{Experimental Results}
\paragraph{}
In this section I will briefly describe the experiment and its aims, then display the results and finally analyze and compare the results.

\subsection{Binary Tree Performance by Length and Time}
\paragraph{}
The creation of the trees used in the HLBL model can be done completely separately from the model. We can look at the trees themselves and see if any of their traits contribute to their performance within the HLBL model.  In table \ref{tab:trees}, we can see the tree creation statistics. I will not describe tree $T_6$ in this section, as the tree is described in section \ref{sec:ADAPTIVEQ}.
 
\paragraph{}
The average code length gives us the average number of decisions that must be made per word. The training time is linear in the average code length. As we can see, the Huffman tree has the shortest average length, which is expected. But, surprisingly, the Recursive ADAPTIVE(2,Huffman) tree has a much larger average length. Hopefully this implies that the ADAPTIVE(2,Huffman) tree is giving up short codes for the addition of more useful decision nodes. If we delve further into the code lengths, we can see how various code lengths affect the test word perplexities. In table \ref{tab:perplexityCodeLength}, we can see the breakdown of average word perplexities, of words with code length $n$, for each tree. The breakdown follows the intuition that longer codes will have worse perplexities. This is due to the increased number of decisions that must be made correctly as the code gets longer. The random tree does not stay true to this intuition, though this is perhaps because of the random tree's tendency to have subtrees that look left-deep or right-deep , whereas the other trees tend to be bushy. Figure \ref{fig:deepvsbushytrees} gives an example of left-deep, right-deep and bushy trees. The left-deep or right-deep trees have a much higher probability of going one direction than the other, so long codes may still have high probabilities. 

\begin{figure}\centering
\begin{tabular}{ccc}
\Tree [.x     [.x     [.x    {x} {\dots}  ] {\dots}  ] {\dots}  ] &
\Tree [.x   [.x [.x {x} {x} ] [.x {x} {x} ]  ]    [.x [.x {x} {x} ]  [.x {x} {x} ] ] ] &
\Tree [.x    {\dots}  [.x    {\dots}  [.x    {\dots} {x} ]  ]  ] 
\end{tabular} 
\caption{Left-deep, Bushy, and Right-deep trees}
\label{fig:deepvsbushytrees}
\end{figure}
f
\paragraph{}
All of the trees have as many or almost as many internal nodes as words, meaning we have relatively full trees. A full tree is a tree in which every node, excluding leaf nodes, has two children. The fullness of the trees means that there are not wasted internal nodes that are not on the path to a word. Note that fullness is not the same as bushiness. The varying shapes of the trees (bushiness) accounts for the discrepancies in the average code length. The training time is also linear in the number of non-leaf nodes, as they correlate directly to the $R$ matrix.
\paragraph{}
We see the greatest differences in the trees in the time it takes to build them. The random tree only depends on the vocabulary so its build time has order $O(|V|)$. Oddly, in table \ref{tab:trees}, the random tree takes longer to build than the Huffman tree. The Huffman tree has creation order $O(|V|\times log|V)|$ and should be slower to build than the random tree.  The discrepancy is due to poor handling of random nodes in my code causing random trees to be built slowly. The Brown Clustering tree is by far the most time intensive tree to build. The Brown Clustering tree takes order $O(|V|^3+n|)$, where $n$ is the corpus length. It is slower to build than the Recursive ADAPTIVE($r,t$) trees which requires $r-1$ trees and $r-1$ models to be built and trained. The Recursive ADAPTIVE($r,t$) tree has creation order $O ( |V| \times log|V| \times g + (r-1)\times m + (r-1)\times t)$ where $g$ is the number of iterations it takes the Gaussian Mixture Model to converge, and $m$ is the maximum training complexity of one of the previous recursive HLBL models.

\begin{table*} \centering
\ra{1.3}
\begin{tabular}{@{}C{1.5cm} c C{1.5cm} C{2cm} C{2cm}@{}}\toprule 
Label & Algorithm & Mean Code Length & Non-Leaf Node Count  & Time to Build Tree\\ 
\midrule
$T_1$ & Random & 29.147 & 10530 & 1.2s\\
$T_2$ & Huffman & 26.257 & 10530 & 0.1s\\
$T_3$ & Brown Cluster & 28.388 & 10528 &271600.39s \\
$T_4$ & Recursive ADAPTIVE (3,Random)& 28.218 & 10531 &16735.63s \\
$T_5$ & Recursive ADAPTIVE (2,Huffman)& 32.254 & 10531 &10514.82s\\
$T_6$ & ADAPTIVE(word2vec) & 25.82& 10530& 2067.53s\\
\bottomrule
\end{tabular}
\caption{Trees for HLBL model. Recursive ADAPTIVE $(n,tree)$ means it was an ADAPTIVE tree recursively run $n$ times and with an initial HLBL model using $tree$. The ADAPTIVE($x$) tree is an ADAPTIVE tree that was built by clustering together $x$ representations.}
\label{tab:trees}
\end{table*}

\begin{table*} \centering
\ra{1.3}
\begin{tabular}{@{}cccccc@{}}\toprule
Code Length & $T_1$ & $T_2$ & $T_3$ & $T_4$ & $T_5$ \\ \midrule
 3 	&           	  &         	   &  7.26        		 &                 		&                	\\
 4 	&  1.0      	  & 9.32    	   &  20.94       		 &                 		&                	\\
 5 	&  1120.88  	  & 7.19    	   &  32.29       		 &                 		&                	\\
 6 	&  453.48   	  & 24.38   	   &  63.8864     		 &   244.59        		&   1.0          	\\
 7 	&  40.52    	  & 29.15   	   &  239.025     		 &   106.42        		&   145.95       	\\
 8 	&  13873.40 	  & 59.73   	   &  363.81      		 &   253.24        		&   303.32       	\\
 9 	&  4009.30  	  & 113.15  	   &  648.24      		 &   884.43        		&   426.96       	\\
 10	&  11439.68 	  & 256.25  	   &  1401.81     		 &   974.86        		&   1042.68      	\\
 11	&  9299.78  	  & 606.39  	   &  3044.13     		 &   1886.32       		&   2061.74      	\\
 12	&  10495.94 	  & 1377.36 	   &  6088.63     		 &   3904.36       		&   3434.53      	\\
 13	&  251634.94	  & 3645.85 	   &  12131.10    		 &   7362.55       		&   2360.03      	\\
 14	&  122953.48	  & 26992.3 	   &  25847.13    		 &   13369.65      		&   5931.19      	\\
 15	&  81351.98 	  & 22990.0 	   &  40362.92    		 &   27488.21      		&   11402.56     	\\
 16	&  18468.03 	  & 19128.9 	   &  62722.05    		 &   53891.50      		&   38567.71     	\\
 17	&  23142.77 	  & 31569.1 	   &  106630.79   		 &   116346.75     		&   58304.71     	\\
 18	&  17788.92 	  & 1.0     	   &              		 &   225282.24     		&   103076.19    	\\
 19	&  26641.27 	  &         	   &              		 &   461424.90     		&   153507.21    	\\
 20	&  42511.83 	  &         	   &              		 &   1075008.64    		&   1117195.33   	\\
 21	&  64427.15 	  &         	   &              		 &   2251914.54    		&   928355.43    	\\
 22	&  19255.16 	  &         	   &              		 &   5487420.11    		&   2653912.58   	\\
 23	&  35200.19 	  &         	   &              		 &   9891214.67    		&   7303232.64   	\\
 24	&  40986.71 	  &         	   &              		 &   25164936.17   		&   14105262.04  	\\
 25	&  11930.46 	  &         	   &              		 &   49709006.56   		&   33638785.96  	\\
 26	&  18254.02 	  &         	   &              		 &   152108176.37  		&   72852492.39  	\\
 27	&  7165.78  	  &         	   &              		 &   231516402.92  		&   246464544.21 	\\
 28	&  10265.89 	  &         	   &              		 &   168746697.15  		&   584429056.52 	\\
 29	&  5248.084 	  &         	   &              		 &   744083147.85  		&   1007277159.20	\\
 30	&  14582.42 	  &         	   &              		 &   1445162704.19 		&   2197227644.04	\\
 31	&  15846.44 	  &         	   &              		 &   2641639930.40 		&   2742412857.71	\\
 32	&  127.89   	  &         	   &              		 &   1535355556.44 		&   3348260000.0 	\\
 33	&           	  &         	   &              		 &   1.0           		&                	\\
\bottomrule
\end{tabular}
\caption{Average perplexity per code length}
\label{tab:perplexityCodeLength}
\end{table*}

\subsection{Recursive Adaptive Trees}
\paragraph{}
I ran an experiment to see how many levels of recursion for the Recursive Adaptive Trees was optimal. The results for the Recursive Adaptive tree with an initial random tree are in table \ref{tab:recursiveAdaptive}. The best level of recursion was 3 with the initial random tree. It is interesting to look at the code length as the number of recursion levels go up. We see that initially, the tree has a very low code length, but the average tree length jumps up sharply, on the next iteration. This correlates to a sharp drop in perplexity. I imagine the tree must be reorganizing itself with more internal nodes to help make more natural decisions. We see that after recursion level 2, the average tree length starts going down again. I assume this means the opposite trade-off is now happening, in return for fewer decisions overall, the tree gets rid of potentially useful decision nodes. I also ran the same experiment with an initial Huffman tree and an initial Brown Cluster tree. The Recursive ADAPTIVE Huffman tree's best perplexity was on recursion level 2. The Brown Cluster on the hand did not perform better with any level of recursion. I think this implies that the Brown Clustering already has good decision nodes (and good clusters), and that reorganizing the tree actually detracts from the original decision nodes.
\begin{table*} \centering
\ra{1.3}
\begin{tabular}{@{}cccc@{}}\toprule
Recursion Level & Perplexity & Code Length\\ 
\midrule
$1$ & 118.596 & 25.70\\
$2$ & 105.981 & 33.03\\
$3$ & 104.583 & 28.22\\
$4$ & 110.969 & 25.61\\
$5$ & 109.577 & 25.39\\
\bottomrule
\end{tabular}
\caption{The effect of more recursion levels on the Recursive ADAPTIVE tree with an initial random tree}
\label{tab:recursiveAdaptive}
\end{table*}

\subsection{Varying Representation Dimensionality}
\paragraph{}
The expressivity of the model directly depends on the representation dimensionality. As the size of the vocabulary and training set grows, the expressivity of the model should also grow in order to capture the additional information given by new word combinations and patterns. Mnih and Hinton in \cite{MnihHinton2009} explore varying representation dimensionalities under 100 dimensions. They found that larger dimensions afford better model performance. Mnih and Hinton concluded that 100 dimensions was enough to capture the complexities of the APNews dataset's 17964K vocabulary. I looked at representation dimensionalities larger than 100, and found that larger word representations for the HLBL model trained on the WSJ dataset helped the model performance. The results can be found in table \ref{tab:wordRepSize}. Simply by increasing the representation size, even for the HLBL model with a random tree, we can see an increase in performance. The increases seem to be tailing off around 300 dimensions.  The HLBL model might need larger representation sizes than the standard LBL model, as the node representations may need to be larger to capture the complexities of the decision tree.


\begin{table*} \centering
\ra{1.3}
\begin{tabular}{@{}cccc@{}}\toprule
Word Representation Size & Perplexity using Random Tree\\ 
\midrule
$100$ & 119.339 \\
$150$ & 118.483 \\
$200$ & 113.472 \\
$250$ & 112.519 \\
$300$ & 111.102 \\
\bottomrule
\end{tabular}
\caption{The effect of word representation size on test perplexity.}
\label{tab:wordRepSize}
\end{table*}

\subsection{Comparison Against Other Models}
\paragraph{}
I also compared the HLBL model run with the various trees from table \ref{tab:trees} to other language models. All of the HLBL models had the same parameters as described in section \ref{sec:experiments}. The LBL and Factored LBL models were implemented by Phil Blunsom, et al. The factored LBL model is the frequency binned model described in section \ref{sec:frequencybinning}. The model uses 100 frequency classes. The 5-gram with modified Kneser Ney smoothing was run using the SRILM implementation \cite{Alumae2010}. The models were allowed to run until their test perplexities stopped decreasing. The results are given by table \ref{tab:languageModelComparison}. The reduction in perplexities and entropy is given by table \ref{tab:reductionPerplexity}. 

\paragraph{}
All of the HLBL models, including using a random tree, perform better than the 5-gram Kneser Ney model,  so we see that even simple neural models are more expressive than co-occurence statistics. The best proposed HLBL model uses a Brown Cluster tree. The HLBL model with a Brown Cluster tree has a 17.24\% reduction in perplexity and a 3.93\% reduction in entropy over the 5-gram model and is not that much worse in terms of performance than the LBL model. However, creating the Brown Cluster tree is incredibly slow, and overall the total time for the model is $58\times$ slower than the LBL model. While the Brown Cluster tree performs as well as any other tree in terms of training time per epoch and testing time, the tree creation time is extremely high. As the overall purpose of this project is to find a tree that is both fast and performs reasonably, the Brown Cluster tree, sadly, does not cut it. Probably the best tree given these criteria, is the Huffman tree. The HLBL model with the Huffman tree is $2.32\times$ faster than the LBL model but 16.19\% worse in terms of perplexity and is 3.71\% worse in terms of entropy. Still, this can be a small price to pay when it comes to very large corpora with hundreds of thousands of words in their vocabulary for a $O(\frac{|V|}{log|V|})$ speed-up. The last two HLBL models use the Recursive ADAPTIVE algorithm. The Recursive ADAPIVE algorithm essentially creates Brown Cluster like trees given existing representations. The resultant trees have semantically/syntactically similar words close together in the tree. At least on a vocabulary of 10K, the Recursive ADAPTIVE algorithm is faster than the Brown Clustering algorithm at creating trees. The Recursive ADAPTIVE algorithm is not fast enough however to be worth using over the simple Huffman tree. 

\begin{table*} \centering
\ra{1.3}
\begin{tabular}{@{}ccccC{2cm}cc@{}}\toprule
Model & Tree & Perplexity & Epochs & Training Time Per Epoch & Testing Time & Total Time\\ 
\midrule
 HLBL & $T_1$ &119.339 & 71& 40.19s & 0.57s& 2855.26s\\
 HLBL & $T_2$ &115.982 & 99& 28.93s & 0.47s & 2864.64s\\
 HLBL & $T_3$ &101.563 & 118& 28.98s & 0.49s & 275020.52s \\
 HLBL & $T_4$ &104.583 & 122& 32.48s & 0.52s & 20698.71s\\
 HLBL & $T_5$ &106.992 & 114& 37.08s & 0.58s& 14742.52\\
 LBL& - &97.21 &62 &106.98s & 21.47s& 6654.23s\\
 Factored LBL & - &98.42&101 & 47.19s & 0.56s & 4766.75s \\
 5-gram KN & - &122.752& 1 & 5.51s & 0.348s & 5.51s\\
\bottomrule
\end{tabular}
\caption{Comparison of HLBL model with various trees and other language models on WSJ dataset}
\label{tab:languageModelComparison}
\end{table*}

\begin{table*} \centering
\ra{1.3}
\begin{tabular}{@{}cccc@{}}\toprule
Model & Tree & Reduction in Perplexity & Reduction in Entropy\\ 
\midrule
 HLBL & $T_1$ &2.78\% & 0.58\%\\
 HLBL & $T_2$ &5.12\% & 1.17\%\\
 HLBL & $T_3$ &17.24\% & 3.93\% \\
 HLBL & $T_4$ &14.8\% & 3.32\%\\
 HLBL & $T_5$ &12.84\% & 2.85\%\\
 LBL& - & 20.81\% &4.84\% \\
 Factored LBL & - &19.82\%&4.59\%\\
\bottomrule
\end{tabular}
\caption{Reduction in perplexity and entropy compared to the 5-gram Kneser-Ney smoothed model on WSJ dataset}
\label{tab:reductionPerplexity}
\end{table*}

\subsection{Visualization} \label{sec:tsne}
\paragraph{}
To see what kinds of word representations are learned by the HLBL models, I use the t-SNE visualization technique described in \cite{Maaten2008} and implemented by Laurens van der Maaten. Using t-SNE, I visualized each context word representation in matrix $R$ on a two-dimensional map. The context word is displayed on top of the two-dimensional coordinates. This should give us an idea of the types of similar words that are learned. Additionally, I use t-SNE to visualize words learned by the tree in the matrix $Q$. Since $Q$ vectors do not directly correspond to words, I use the $Q$ node directly above a leaf word or leaf words to represent the word, or word-pair. After t-SNE is applied to the $Q$ vectors, the word or word-pair is then displayed on top of the two-dimensional location representing the $Q$ vector. The factored model's $R$ and $Q$ matrix are displayed as a baseline.
\paragraph{}
As the full visualization over all 10K vocabulary words is quite large, I have selected some areas from the Q word cloud in figure \ref{fig:Qcloud} and some areas for the R word cloud in figure \ref{fig:Rcloud}. A fuller version of the visualizations can be found in Appendix B. 
\paragraph{}
I found the HLBL models, using the random tree and the Huffman tree, learned some Q word representations that made semantic or syntactic sense, whereas the models with the other three trees did not. In figure \ref{fig:Qcloud} we can see that the HLBL model with the random tree clustered together numbers in the first example, and pronouns and prepositions in the second example. The HLBL model with the Huffman tree clustered together names and places. I think this may be because the random and Huffman tree HLBL models are forced to learn word "meanings" since the tree does not automatically group together similar words. For the other three models, the trees separated out the word meanings, and so instead of focusing on learning the word meanings, the model then focused on learning the decision paths. The other three models learned word representations that are optimized for correctly predicting the path through the tree. Since the other three models have better perplexities than the random and Huffman tree HLBL models, I assume this is because the other three trees start off with better word clusterings than is learned by the HLBL model with random or Huffman trees. This is supported by the fact that the Q word clouds are still relatively uniform looking and unclustered (with some exceptions), for the random and Huffman HLBL models. Additionally, the other three models have the benefit of learning the paths better.
\paragraph{}
I found the same pattern as the Q matrices also held for the R matrices. We see that the HLBL models with the random and Huffman trees seem to learn some semantic or syntactic patterns whereas the other three trees did not. The two examples in figure \ref{fig:Rcloud} for the random tree have clustered together numbers and have clustered together names. The Huffman tree seems to have learned verbs and prepositions. In the first example for the Huffman tree, it looks like the left word is a present tense verb, and the right word is a past tense verb, with the exception of 'pressing'. I think the reason behind the random and Huffman tree HLBL models learning semantic and syntactic patterns for the R matrix is the same as the reasoning behind the models learning such patterns for the Q matrix. Since the Q and R matrices are learned jointly, it makes sense that they both show the same emphasis in learning. Another observation is that the Brown Cluster tree and the Recursive Adaptive trees have extremely round Q word-clouds. This might mean that the Q matrices for these three trees are close to multi-dimensional Gaussians, and do not help very much with the models.

\begin{figure}[p]
\centering
\begin{tabular}{@{}m{2cm}ccc@{}}
$T_1$ &
\includegraphics[width=0.15\textheight]{./images/tsne/Q_random_it71_thumb.png} &
\includegraphics[width=0.15\textheight]{./images/tsne/Q_random_it71_small1.png} &
\includegraphics[width=0.15\textheight]{./images/tsne/Q_random_it71_small2.png}
\\
$T_2$ &
\includegraphics[width=0.15\textheight]{./images/tsne/Q_Huff_iter132_thumb.png} &
\includegraphics[width=0.15\textheight]{./images/tsne/Q_Huff_iter132_small1.png} &
\includegraphics[width=0.15\textheight]{./images/tsne/Q_Huff_iter132_small2.png}
\\
$T_3$ &
\includegraphics[width=0.15\textheight]{./images/tsne/Q_Brown_iter_118_thumb.png} &
\includegraphics[width=0.15\textheight]{./images/tsne/Q_Brown_iter_118_small1.png} &
\includegraphics[width=0.15\textheight]{./images/tsne/Q_Brown_iter_118_small2.png}
\\
$T_4$ &
\includegraphics[width=0.15\textheight]{./images/tsne/Q_adaptiveR3_thumb.png} &
\includegraphics[width=0.15\textheight]{./images/tsne/Q_adaptiveR3_small1.png} &
\includegraphics[width=0.15\textheight]{./images/tsne/Q_adaptiveR3_small2.png}
\\
$T_5$ &
\includegraphics[width=0.15\textheight]{./images/tsne/Q_adaptive-huffR1_it114_thumb.png} &
\includegraphics[width=0.15\textheight]{./images/tsne/Q_adaptive-huffR1_it114_small1.png} &
\includegraphics[width=0.15\textheight]{./images/tsne/Q_adaptive-huffR1_it114_small2.png}
\\
Factored Model &
\includegraphics[width=0.15\textheight]{./images/tsne/Q_factored_it7_step0_05_thumb.png} &
\includegraphics[width=0.15\textheight]{./images/tsne/Q_factored_it7_step0_05_small1.png} &
\includegraphics[width=0.15\textheight]{./images/tsne/Q_factored_it7_step0_05_small2.png}
\end{tabular}
\caption{Q matrices for various trees projected into two dimensions. The left-most image shows the general shape of all of the words together, and the center and right-most image show close-ups of the word cloud.}
\label{fig:Qcloud}
\end{figure}

\begin{figure}[p]
\centering
\begin{tabular}{@{}m{2cm}ccc@{}}
$T_1$ &
\includegraphics[width=0.15\textheight]{./images/tsne/R_random_it71_thumb.png} &
\includegraphics[width=0.15\textheight]{./images/tsne/R_random_it71_small1.png} &
\includegraphics[width=0.15\textheight]{./images/tsne/R_random_it71_small2.png}
\\
$T_2$ &
\includegraphics[width=0.15\textheight]{./images/tsne/R_Huff_iter132_thumb.png} &
\includegraphics[width=0.15\textheight]{./images/tsne/R_Huff_iter132_small1.png} &
\includegraphics[width=0.15\textheight]{./images/tsne/R_Huff_iter132_small2.png}
\\
$T_3$ &
\includegraphics[width=0.15\textheight]{./images/tsne/R_Brown_iter_118_thumb.png} &
\includegraphics[width=0.15\textheight]{./images/tsne/R_Brown_iter_118_small1.png} &
\includegraphics[width=0.15\textheight]{./images/tsne/R_Brown_iter_118_small2.png}
\\
$T_4$ &
\includegraphics[width=0.15\textheight]{./images/tsne/R_adaptiveR3_thumb.png} &
\includegraphics[width=0.15\textheight]{./images/tsne/R_adaptiveR3_small1.png} &
\includegraphics[width=0.15\textheight]{./images/tsne/R_adaptiveR3_small2.png}
\\
$T_5$ &
\includegraphics[width=0.15\textheight]{./images/tsne/R_adaptive-huffR1_it114_thumb.png} &
\includegraphics[width=0.15\textheight]{./images/tsne/R_adaptive-huffR1_it114_small1.png} &
\includegraphics[width=0.15\textheight]{./images/tsne/R_adaptive-huffR1_it114_small2.png}
\\
Factored Model &
\includegraphics[width=0.15\textheight]{./images/tsne/R_factored_it7_step0_05_thumb.png} &
\includegraphics[width=0.15\textheight]{./images/tsne/R_factored_it7_step0_05_small1.png} &
\includegraphics[width=0.15\textheight]{./images/tsne/R_factored_it7_step0_05_small2.png}
\end{tabular}
\caption{R matrices for various trees projected into two dimensions. The left-most image shows the general shape of all of the words together, and the center and right-most image show close-ups of the word cloud.}
\label{fig:Rcloud}
\end{figure}

\subsection{word2vec Q initialization} \label{sec:ADAPTIVEQ}
\paragraph{}
One other experiment I tried was to use better word representations to initialize $Q$. After observing that the HLBL model with the Brown Cluster tree and the Recursive ADAPTIVE trees do not seem to learn semantically or lexically motivated clusters in their Q matrices, I thought that by initializing the Q matrices better, I could push the models to learn better Q matrices. I decided to use the word2vec toolkit representations implemented by Mikolav et al to initialize my $Q$ matrix. The word2vec toolkit is based off of the ideas in \cite{Mikolov2013}. I created word2vec representations with a skip-gram model, the hierarchical soft-max, a word-window of 5, and a representation dimensionality of 100. The representation creation took 2.309 seconds running with 12 threads for the WSJ dataset. The results for the HLBL model with the Brown Cluster tree and word2vec $Q$ initialization are given by table \ref{tab:brownWord2vec}. We see that with better $Q$ initialization, the HLBL model with a Brown Cluster tree can perform almost as well as the LBL model. The perplexity of the Brown Cluster tree is only $-0.43$ away from the LBL's perplexity. Unfortunately, this still suffers from the long creation time of the Brown Cluster tree.
\paragraph{}
To avoid the long creation time of the Brown Cluster tree, I also tried using an ADAPTIVE tree. The ADAPTIVE tree tries to cluster together similar word representations together in the tree. If the word representations capture semantic and syntactical information, then the ADAPTIVE tree acts much like the Brown Cluster tree. To keep the tree creation time low, I did not run the ADAPTIVE algorithm recursively. Using the same word2vec representations described above, I created an ADAPTIVE tree, and also initialized the Q matrix to the word2vec representations. The results can be seen in table \ref{tab:brownWord2vec}. The ADAPTIVE tree does not perform as well as the Brown Cluster tree, but is significantly faster to create. The ADAPTIVE tree is also faster overall than the LBL model with an allowable loss in performance. 

\paragraph{}
I also tried initializing the Q matrix with the word2vec representations for the Huffman tree with no performance gain. The results can be seen in table \ref{tab:brownWord2vec}. I think this may be because the Huffman tree model does a good job learning the context words even without sensible initialization. 

\begin{table*} \centering
\ra{1.3}
\begin{tabular}{cccccc}\toprule
Tree & Perplexity & Epochs & Training Time Per Epoch & Testing Time & Total Time\\ 
\midrule
$T_2$ & 115.786 & 81 & 26.31s &0.46s & 2133.98s \\
$T_3$ & 97.64 & 115& 30.92s & 0.49s& 275022.82s\\
$T_6$& 104.477 & 83& 35.46s & 0.56s& 5013.57s\\
\bottomrule
\end{tabular}
\caption{HLBL model with Q initialized to word2vec representations on WSJ dataset.}
\label{tab:brownWord2vec}
\end{table*}

\subsection{Visualization}
\paragraph{}
I ran the t-SNE visualizations described in section \ref{sec:tsne} over the HLBL models with Q initialized to the word2vec representations. Note that I did not include the Huffman tree with word2vec Q initialization since the model does not seem to be better than its non word2vec initialized counterpart. The results can be seen in figures \ref{fig:RcloudWord2Vec} and {fig:QcloudWord2Vec}. We see that the Q word-clouds make more semantic/syntactical clusters unlike the models without the word2vec Q initialization. This is exactly the result we were looking for. The R word-clouds remain largely unchanged. 

\begin{figure}[p]
\centering
\begin{tabular}{@{}C{2cm}ccc@{}}
$T_3$&
\includegraphics[width=0.15\textheight]{./images/tsne/Q_Brown_QinWord2Vec_thumb.png} &
\includegraphics[width=0.15\textheight]{./images/tsne/Q_Brown_QinWord2Vec_small1.png} &
\includegraphics[width=0.15\textheight]{./images/tsne/Q_Brown_QinWord2Vec_small2.png}
\\
$T_6$ &
\includegraphics[width=0.15\textheight]{./images/tsne/Q_adaptive_QinWord2Vec_RinWord2Vec_thumb.png} &
\includegraphics[width=0.15\textheight]{./images/tsne/Q_adaptive_QinWord2Vec_RinWord2Vec_small1.png} &
\includegraphics[width=0.15\textheight]{./images/tsne/Q_adaptive_QinWord2Vec_RinWord2Vec_small2.png}
\end{tabular}
\caption{Q matrices for HLBL models with Q initialized to word2vec representations The left-most image shows the general shape of all of the words together, and the center and right-most image show close-ups of the word cloud.}
\label{fig:QcloudWord2Vec}
\end{figure}

\begin{figure}[p]
\centering
\begin{tabular}{@{}C{2cm}ccc@{}}
$T_2$ &
\includegraphics[width=0.15\textheight]{./images/tsne/R_Brown_QinWord2Vec_thumb.png} &
\includegraphics[width=0.15\textheight]{./images/tsne/R_Brown_QinWord2Vec_small1.png} &
\includegraphics[width=0.15\textheight]{./images/tsne/R_Brown_QinWord2Vec_small2.png}
\\
$T_6$ &
\includegraphics[width=0.15\textheight]{./images/tsne/R_adaptive_QinWord2Vec_RinWord2Vec_thumb.png} &
\includegraphics[width=0.15\textheight]{./images/tsne/R_adaptive_QinWord2Vec_RinWord2Vec_small1.png} &
\includegraphics[width=0.15\textheight]{./images/tsne/R_adaptive_QinWord2Vec_RinWord2Vec_small2.png}
\end{tabular}
\caption{R matrices for HLBL models with Q initialized to word2vec representations The left-most image shows the general shape of all of the words together, and the center and right-most image show close-ups of the word cloud.}
\label{fig:RcloudWord2Vec}
\end{figure}


\section{Large Vocabulary}
\paragraph{}
We see that with very large vocabulary, the HLBL model really pulls ahead in terms of speed. I ran some experiments with the 85K vocabulary BNC dataset.  I ran the HLBL model with two different trees, the Huffman tree and also the ADAPTIVE(word2vec) tree with Q initialized to the word2vec representations. The BNC word2vec representations took 4.960s to create running with 12 threads. 
The factored model that performed very well in terms of perplexity and speed on the 10K WSJ vocabulary, performs much slower with the 85K BNC vocabulary. The factored model's training and testing time scales linearly with the vocabulary size. The LBL model's training and testing time also scales linearly with the vocabulary size (though with a larger scaling constant). The HLBL model on the other hand scales logarithmically. In table \ref{tab:largeVocabulary}, it becomes obvious why speed up techniques are necessary for computing language models over real language data.The HLBL model with the Huffman tree is \hl{$X\times$ faster to compute than the LBL model. Though there is a performance loss of $X\%$.} The HLBL model with the ADAPTIVE(word2vec) tree and Q initialized to word2vec representations is \hl{$X\times$ faster to compute than the LBL model and has a performance loss of $X\%$.} This makes the HLBL model with the ADAPTIVE tree and word2vec Q initialization the best model overall, of the models I explored.

%\begin{table*} \centering
%\ra{1.3}
%\begin{tabular}{@{}ccC{2cm}cc@{}}\toprule
%Model & Perplexity & Training Time Per Epoch & Testing Time & Total Time\\ 
%\midrule
% LBL&105.442 &111.05s & 21.47s & 6906.57s \\
% Factored LBL &108.802& 71.05s & 0.60s & 7887.15s\\
% HLBL with Huffman Tree &131.913& 33.26s &0.48s &3492.79s \\
%\bottomrule
%\end{tabular}
%\caption{The effect of large vocabularies on training and testing times}
%\label{tab:largeVocabulary}
%\end{table*}

\begin{table*} \centering
\ra{1.3}
\begin{tabular}{@{}C{2cm}ccccC{2cm}C{2cm}C{2cm}@{}}\toprule
Model & Tree & Q init & Perplexity & Epochs & Training Time Per Epoch & Testing Time & Total Time\\ 
\midrule
 LBL& - & Gaussian & ? &? &?s& ?s & ? \\
 Factored LBL & - & Gaussian &144.32& 112 & 479.37s & 72.14s & 53761.58s\\
 HLBL& $T_2$ & Gaussian &182.06& 117& 102.7s &30.18s &12047.61s \\
 HLBL & $T_6$ & word2vec &159.187& 106& 129.81s &38.32s &13803.14s \\
\bottomrule
\end{tabular}
\caption{The effect of a large vocabulary (85K) from the BNC dataset on training and testing times}
\label{tab:largeVocabulary}
\end{table*}
  
  
  

\chapter{Experiments and Results}
In this chapter I will define the experiments and state the experimental results. I will then analyze the results and their implications.

\section{Experiments} \label{sec:experiments}
\paragraph{}
I have conducted experiments to show the performance and speed of the HLBL model with various trees. To be able to compare the models, I use the same parameter values for each model. I use a word and node representation size of 100 dimensions, a $5$-word context (including the target word), a mini batch size of 10,000 and a learning step-size of 0.01. I also use a small regularization constant while training all of the models. Most of the models can be trained faster if the parameters are tuned to the individual model. The models are all trained on the same machine using a single thread. The machine runs at 3.47GHz and has 60GB of RAM.

\subsection{Dataset}
\paragraph{}
For most of my experiments, I use the Penn Treebank dataset. Specifically, I use the Wall Street Journal (WSJ) sections 2-21 for my training set and section 22 for my testing set. There are 950,028 training tokens and 40,117 testing tokens. The training set has been preprocessed to replace any word that occurs less than 6 times with "\_UNK\_". This reduces the vocabulary to 10,531 words. The testing set has also been preprocessed to replace any word not seen in the training set with "\_UNK\_".

%WSJ sections 2-21, with words that occur only once preprocessed out, as a training set. This set has a vocabulary size of 23768. 

%\subsection{Varying Trees}
%I use random Trees, Huffman trees, Brown Cluster trees, ADAPTIVE trees, and Recursive Adaptive Trees with the HLBL model.

\section{Experimental Results}
\paragraph{}
In this section I will briefly describe the experiments and their aims, then display the results and finally analyze and compare the results.

\subsection{Binary Tree Performance by Length and Time}
\paragraph{}
The creation of the trees used in the HLBL model can be done separately from the model. We can look at the trees themselves and see if any of their traits contribute to their performance within the HLBL model.  In Table \ref{tab:trees}, we can see the tree creation statistics. I will not describe tree $T_6$ in this section, as the tree is described in Section \ref{sec:ADAPTIVEQ}.
 
\paragraph{}
The mean code length gives us the average number of decisions that must be made per word. The training time is linear in the mean code length. We would expect the Huffman tree to have the shortest mean length, but the ADAPTIVE(word2vec) tree is actually shorter. This is because the Huffman tree is guaranteed to have the shortest average mean length on the training data unigram frequencies, but the ADAPTIVE(word2vec) tree is created using frequencies learned from the testing data. The Huffman tree has the shortest mean length when the word frequencies agree with those used to create the tree. Given that the Huffman tree has the shortest mean length for the training set data, it is interesting that the Recursive~ADAPTIVE(2,Huffman) tree has a much larger mean length. Hopefully this implies that the Recursive~ADAPTIVE(2,Huffman) tree is giving up short codes for the addition of more useful decision nodes. If we delve further into the code length, we can see how code length affects the test word perplexities. In Table \ref{tab:perplexityCodeLength}, we can see the breakdown of average word perplexities per code length for each tree. The breakdown follows the intuition that longer codes will have worse perplexities. This is due to the increased number of decisions that must be made correctly as the code gets longer. The random tree does not stay true to this intuition, though this is perhaps because of the random tree's tendency to have subtrees that look left-deep or right-deep , whereas the other trees tend to be bushy. Figure \ref{fig:deepvsbushytrees} gives an example of left-deep, bushy, and right-deep trees. The left-deep and right-deep trees have much higher probabilities of going one direction than the other, so long codes may still have high probabilities. The left-deep and right-deep tendencies of the random tree are artifacts of my implementation. In general, random trees should be fairly balanced.

\begin{figure}\centering
\begin{tabular}{ccc}
\Tree [.x     [.x     [.x    {x} {\dots}  ] {\dots}  ] {\dots}  ] &
\Tree [.x   [.x [.x {x} {x} ] [.x {x} {x} ]  ]    [.x [.x {x} {x} ]  [.x {x} {x} ] ] ] &
\Tree [.x    {\dots}  [.x    {\dots}  [.x    {\dots} {x} ]  ]  ] 
\end{tabular} 
\caption{Left-deep, bushy, and right-deep trees}
\label{fig:deepvsbushytrees}
\end{figure}

\paragraph{}
All of the trees have as many or almost as many internal nodes as words. Since the number of internal nodes is not greater than the number of words, the trees are not creating extra un-useful internal nodes. Additionally, the trees are full. A full tree is a tree in which every node, excluding leaf nodes, has two children. Note that fullness is not the same as bushiness. The varying shapes of the trees~(bushiness) accounts for the discrepancies in the mean code length. The training time is also linear in the number of non-leaf nodes, as they correlate directly to rows of the $R$ matrix.
\paragraph{}
We see the greatest differences in the trees in the time it takes to build them. The random tree only depends on the vocabulary so its build time has order $O(|V|)$. Oddly, in Table \ref{tab:trees}, the random tree takes longer to build than the Huffman tree. The Huffman tree has creation order $O(|V|\times \log|V|)$ and should be slower to build than the random tree.  This discrepancy is due to poor handling of random nodes in my code causing random trees to be built slowly. The~Brown Clustering tree is by far the most time intensive tree to build. The Brown Clustering tree takes order $O(|V|^3+n)$, where $n$ is the corpus length. It is slower to build than the Recursive ADAPTIVE($r,t$) trees which require $r-1$ trees and $r-1$ models to be built and trained. The Recursive ADAPTIVE($r,t$) tree has creation order $O ( |V| \times \log|V| \times g + (r-1)\times m + (r-1)\times t)$ where $g$ is the number of iterations it takes the Gaussian Mixture Model to converge, and $m$ is the maximum training complexity of one of the previous recursive HLBL models.

\begin{table*} \centering
\ra{1.3}
\begin{tabular}{@{}C{1.5cm} c C{1.5cm} C{2cm} C{2cm}@{}}\toprule 
Label & Algorithm & Mean Code Length & Non-Leaf Node Count  & Time to Build Tree\\ 
\midrule
$T_1$ & Random & 29.147 & 10530 & 1.2s\\
$T_2$ & Huffman & 26.257 & 10530 & 0.1s\\
$T_3$ & Brown Cluster & 28.388 & 10528 &271600.39s \\
$T_4$ & Recursive ADAPTIVE(3,Random)& 28.218 & 10531 &16735.63s \\
$T_5$ & Recursive ADAPTIVE(2,Huffman)& 32.254 & 10531 &10514.82s\\
\midrule
$T_6$ & ADAPTIVE(word2vec) & 25.823& 10530& 2067.53s\\
\bottomrule
\end{tabular}
\caption{Trees for HLBL model. Recursive ADAPTIVE$(n,tree)$ means it was an ADAPTIVE tree recursively run $n$ times and with an initial HLBL model using $tree$. The ADAPTIVE($x$) tree is an ADAPTIVE tree that was built by clustering together $x$ representations.}
\label{tab:trees}
\end{table*}

\begin{table*} \centering
\ra{1.3}
\begin{tabular}{@{}cccccc@{}}\toprule
Code Length & $T_1$ & $T_2$ & $T_3$ & $T_4$ & $T_5$ \\ \midrule
 3 	&           	  &         	   &  7.26        		 &                 		&                	\\
 4 	&  1.0      	  & 9.32    	   &  20.94       		 &                 		&                	\\
 5 	&  1120.88  	  & 7.19    	   &  32.29       		 &                 		&                	\\
 6 	&  453.48   	  & 24.38   	   &  63.8864     		 &   244.59        		&   1.0          	\\
 7 	&  40.52    	  & 29.15   	   &  239.025     		 &   106.42        		&   145.95       	\\
 8 	&  13873.40 	  & 59.73   	   &  363.81      		 &   253.24        		&   303.32       	\\
 9 	&  4009.30  	  & 113.15  	   &  648.24      		 &   884.43        		&   426.96       	\\
 10	&  11439.68 	  & 256.25  	   &  1401.81     		 &   974.86        		&   1042.68      	\\
 11	&  9299.78  	  & 606.39  	   &  3044.13     		 &   1886.32       		&   2061.74      	\\
 12	&  10495.94 	  & 1377.36 	   &  6088.63     		 &   3904.36       		&   3434.53      	\\
 13	&  251634.94	  & 3645.85 	   &  12131.10    		 &   7362.55       		&   2360.03      	\\
 14	&  122953.48	  & 26992.3 	   &  25847.13    		 &   13369.65      		&   5931.19      	\\
 15	&  81351.98 	  & 22990.0 	   &  40362.92    		 &   27488.21      		&   11402.56     	\\
 16	&  18468.03 	  & 19128.9 	   &  62722.05    		 &   53891.50      		&   38567.71     	\\
 17	&  23142.77 	  & 31569.1 	   &  106630.79   		 &   116346.75     		&   58304.71     	\\
 18	&  17788.92 	  & 1.0     	   &              		 &   225282.24     		&   103076.19    	\\
 19	&  26641.27 	  &         	   &              		 &   461424.90     		&   153507.21    	\\
 20	&  42511.83 	  &         	   &              		 &   1075008.64    		&   1117195.33   	\\
 21	&  64427.15 	  &         	   &              		 &   2251914.54    		&   928355.43    	\\
 22	&  19255.16 	  &         	   &              		 &   5487420.11    		&   2653912.58   	\\
 23	&  35200.19 	  &         	   &              		 &   9891214.67    		&   7303232.64   	\\
 24	&  40986.71 	  &         	   &              		 &   25164936.17   		&   14105262.04  	\\
 25	&  11930.46 	  &         	   &              		 &   49709006.56   		&   33638785.96  	\\
 26	&  18254.02 	  &         	   &              		 &   152108176.37  		&   72852492.39  	\\
 27	&  7165.78  	  &         	   &              		 &   231516402.92  		&   246464544.21 	\\
 28	&  10265.89 	  &         	   &              		 &   168746697.15  		&   584429056.52 	\\
 29	&  5248.084 	  &         	   &              		 &   744083147.85  		&   1007277159.20	\\
 30	&  14582.42 	  &         	   &              		 &   1445162704.19 		&   2197227644.04	\\
 31	&  15846.44 	  &         	   &              		 &   2641639930.40 		&   2742412857.71	\\
 32	&  127.89   	  &         	   &              		 &   1535355556.44 		&   3348260000.0 	\\
 33	&           	  &         	   &              		 &   1.0           		&                	\\
\bottomrule
\end{tabular}
\caption{Average perplexity per code length. }
\label{tab:perplexityCodeLength}
\end{table*}

\subsection{Recursive Adaptive Trees}
\paragraph{}
I ran an experiment to find the optimal level of recursion for the Recursive Adaptive tree. The results for various levels of recursion on the Recursive Adaptive tree with an initial random tree are in Table \ref{tab:recursiveAdaptive}. The~best level of recursion was three with an initial random tree. It is interesting to look at the code length as the number of recursion levels go up. We see that initially, the tree has a low code length, but the average tree length jumps up sharply, on the next recursion level. This correlates to a sharp drop in perplexity. I imagine the tree must be reorganizing itself with more internal nodes to help make more natural decisions. We see that after recursion level two, the average tree length starts decreasing again. I assume this means the opposite trade-off is now happening, in return for fewer decisions overall, the tree gets rid of potentially useful decision nodes. I also run the same experiment with an initial Huffman tree and an initial Brown Cluster tree. The Recursive ADAPTIVE Huffman tree's best perplexity is on recursion level two. The Brown Cluster does not perform better with any level of recursion. I think this implies that the Brown Clustering already has good decision nodes (and good clusters), and that reorganizing the tree actually detracts from the original decision nodes.
\begin{table*} \centering
\ra{1.3}
\begin{tabular}{@{}cccc@{}}\toprule
Recursion Level & Perplexity & Code Length\\ 
\midrule
$1$ & 118.596 & 25.70\\
$2$ & 105.981 & 33.03\\
$3$ & 104.583 & 28.22\\
$4$ & 110.969 & 25.61\\
$5$ & 109.577 & 25.39\\
\bottomrule
\end{tabular}
\caption{The effect of more recursion levels on the Recursive ADAPTIVE tree with an initial random tree}
\label{tab:recursiveAdaptive}
\end{table*}

\subsection{Varying Representation Dimensionality}
\paragraph{}
The expressivity of the model directly depends on the representation dimensionality. As the size of the vocabulary and training set grows, the expressivity of the model should also grow in order to capture the additional information given by new word combinations and patterns. Mnih and Hinton in \cite{MnihHinton2009} explore varying representation dimensionalities under 100 dimensions. They find that larger dimensions afford better model performance. Mnih and Hinton conclude that 100 dimensions is enough to capture the complexities of the APNews dataset's 17,964 word vocabulary. I looked at representation dimensionalities larger than 100, and found the HLBL model trained on the WSJ dataset benefits from larger word representations. The results can be found in Table \ref{tab:wordRepSize}. Simply by increasing the representation size, even with a random tree, we can see an increase in performance. The increases seem to tail off around 300 dimensions.  The HLBL model may need larger representation sizes than the standard LBL model, as the node representations may need to be larger to capture the complexities of the decision tree.


\begin{table*} \centering
\ra{1.3}
\begin{tabular}{@{}cccc@{}}\toprule
Word Representation Size & Perplexity using Random Tree\\ 
\midrule
$100$ & 119.339 \\
$150$ & 118.483 \\
$200$ & 113.472 \\
$250$ & 112.519 \\
$300$ & 111.102 \\
\bottomrule
\end{tabular}
\caption{The effect of word representation size on test perplexity.}
\label{tab:wordRepSize}
\end{table*}

\subsection{Comparison against Other Models}
\paragraph{}
I also compared the HLBL model with the various trees from Table \ref{tab:trees} to other language models. All of the HLBL models have the same parameters as described in Section \ref{sec:experiments}. The LBL and Factored LBL models are implemented by Phil Blunsom. The Factored LBL model is the frequency-binned model described in Section \ref{sec:frequencybinning} applied to the LBL model. The model uses 100 frequency classes. The 5-gram with modified Kneser-Ney smoothing is run using the SRILM implementation \cite{Alumae2010}.
The models are allowed to run until their test perplexities stop decreasing. The results are given in Table \ref{tab:languageModelComparison}. The reduction in perplexities and entropy is given in Table \ref{tab:reductionPerplexity}. 

\paragraph{}
The Factored LBL model is the best speedup technique to compare the HLBL model to, as both techniques try to limit the effective vocabulary and also speed up both the training and testing components. The Factored LBL has very good performance and is comparable to the LBL.

\paragraph{}
All of the HLBL models, including using a random tree, perform better than the 5-gram modified Kneser-Ney model,  therefore even simple neural models are more expressive than co-occurence statistics. We can see the performance of all the models compared to the 5-gram model in table \ref{tab:reductionPerplexity}. The best proposed HLBL model in terms of performance uses a Brown Cluster tree ($T_3$) and the best HLBL model considering both time and performance uses the Huffman tree ($T_2$). In Table \ref{tab:perplexityAndSpeedup}, we can see the performance of the Brown Cluster tree and Huffman tree against the LBL and Factored LBL. The Brown Cluster tree HLBL model is pretty good in terms of performance compared to the LBL and Factored LBL models. However, creating the Brown Cluster tree is incredibly slow, and the total time for the HLBL model with a Brown Cluster tree is over an order of magnitude greater than the time for the LBL and Factored LBL models. While the Brown Cluster tree performs as well as any other tree in terms of training time per epoch and testing time, the tree creation time is extremely high. As the overall purpose of this project is to find a tree that is both fast and performs reasonably, the Brown Cluster tree, sadly, does not cut it. Probably the best tree given these criteria, is the Huffman tree. The HLBL model with the Huffman tree is about twice as fast as both the LBL and Factored LBL model, but is also about 15\% worse in terms of perplexity compared to both the LBL and Factored LBL. Still, this can be a price worth paying when it comes to large corpora with hundreds of thousands of words in their vocabulary for a $O(\frac{|V|}{\log|V|})$ speedup versus a $O(\sqrt{|V|})$ speedup. The last two HLBL models use the Recursive ADAPTIVE algorithm ($T_4$ and $T_5$). The Recursive ADAPTIVE algorithm essentially creates Brown Cluster-like trees given existing representations. The resultant trees have semantically/syntactically similar words close together in the tree. At least on a vocabulary of 10K, the Recursive ADAPTIVE algorithm is faster than the Brown Clustering algorithm at creating trees. The Recursive ADAPTIVE tree HLBL models' performances fall between the Brown Cluster and Huffman tree HLBL performances. The Recursive ADAPTIVE algorithm is not fast enough however to be worth using over the simple Huffman tree. 

\begin{table*} \centering
\ra{1.3}
\begin{tabular}{@{}C{3cm}ccC{2cm}C{2cm}cc@{}}\toprule
Model & Tree & Perplexity & Epochs Till Convergence & Training Time Per Epoch & Testing Time & Total Time\\ 
\midrule
 HLBL & $T_1$ &119.339 & 71& 40.19s & 0.57s& 2855.26s\\
 HLBL & $T_2$ &115.982 & 99& 28.93s & 0.47s & 2864.64s\\
 HLBL & $T_3$ &101.563 & 118& 28.98s & 0.49s & 275020.52s \\
 HLBL & $T_4$ &104.583 & 122& 32.48s & 0.52s & 20698.71s\\
 HLBL & $T_5$ &106.992 & 114& 37.08s & 0.58s& 14742.52\\
 LBL& - &97.21 &62 &106.98s & 21.47s& 6654.23s\\
 Factored LBL & - &98.42&101 & 47.19s & 0.56s & 4766.75s \\
 5-gram modified KN & - &122.752& 1 & 5.51s & 0.348s & 5.51s\\
\bottomrule
\end{tabular}
\caption{Comparison of HLBL model with various trees and other language models on WSJ dataset}
\label{tab:languageModelComparison}
\end{table*}

\begin{table*} \centering
\ra{1.3}
\begin{tabular}{@{}ccC{2cm}C{2cm}@{}}\toprule
Model & Tree & Reduction in Perplexity & Reduction in Entropy \\
\midrule
 HLBL & $T_1$ &2.78\% & 0.58\%\\
 HLBL & $T_2$ &5.12\% & 1.17\% \\
 HLBL & $T_3$ &17.24\% & 3.93\%  \\
 HLBL & $T_4$ &14.8\% & 3.32\%\\
 HLBL & $T_5$ &12.84\% & 2.85\%\\
 LBL& - & 20.81\% &4.84\%\\
 Factored LBL & - &19.82\%&4.59\% \\
\bottomrule
\end{tabular}
\caption{Reduction in perplexity and entropy compared to the 5-gram modified Kneser-Ney smoothed model on WSJ dataset.}
\label{tab:reductionPerplexity}
\end{table*}

\begin{table*} \centering
\ra{1.3}
\begin{tabular}{@{}ccC{1.75cm}C{1.75cm}C{1.75cm}C{1.75cm}C{1.75cm}C{1.75cm}@{}}\toprule
Model & Tree
& Increase in Perplexity Compared to LBL & Increase in Entropy Compared to LBL
& Increase in Perplexity Compared to Factored LBL & Increase in Entropy Compared to Factored LBL
& Time Compared to Factored LBL & Time Compared to Factored LBL
\\
\midrule
 HLBL & $T_2$ & 16.19\% & 3.71\% & 15.4\% & 3.45\% & $2.32\times$ faster & $1.66\times$ faster\\
 HLBL & $T_3$ & 4.28\% & 0.95\% & 3.09\% & 0.68\% &$41\times$ slower & $57\times$ slower\\
\bottomrule
\end{tabular}
\caption{HLBL models compared against the LBL and Factored LBL models on the WSJ dataset.}
\label{tab:perplexityAndSpeedup}
\end{table*}


\subsection{Visualization} \label{sec:tsne}
\paragraph{}
To see what kinds of word representations are learned by the HLBL models, I use the t-SNE visualization technique described in \cite{Maaten2008} and implemented by Laurens van der Maaten. Using t-SNE, I visualized each context word representation in matrix $R$ on a two-dimensional map. The context word is displayed on top of the two-dimensional coordinates. This should give us an idea of the types of word similarities that are learned. Additionally, I use t-SNE to visualize words learned by the tree in the matrix $Q$. Since $Q$ vectors do not directly correspond to words, I use the $Q$ node directly above a leaf word or leaf words to represent the word, or word-pair. After t-SNE is applied to the $Q$ vectors, the word or word-pair is then displayed on top of the two-dimensional location representing the $Q$ vector.
\paragraph{}
As the full visualization over all 10K vocabulary words is large, I have selected some areas from the $Q$ word-cloud in Figure \ref{fig:Qcloud} and some areas from the $R$ word-cloud in Figure \ref{fig:Rcloud}. The Factored LBL model's $R$ and $Q$ matrix are displayed as a baseline. A fuller version of the visualizations can be found in Appendix B. 
\paragraph{}
The $Q$ matrix holds the learned vector representations for the context words. I found that the HLBL models, using the random tree ($T_1$) and the Huffman tree ($T_2$), learn some $Q$ word representations that make semantic or syntactic sense, whereas the models with the other three trees ($T_3,T_4,T_5$) do not. In Figure \ref{fig:Qcloud} we can see that the HLBL model with the random tree clusters together numbers in the first example, and pronouns and prepositions in the second example. The HLBL model with the Huffman tree clusters together names and places. I think this may be because the random and Huffman tree HLBL models are forced to learn word "meanings" as the tree does not automatically group together similar words. For the other three models, the trees already separate out the word meanings, and so instead of focusing on learning the word meanings, the model focuses on learning the decision paths. The other three models learn word representations that are optimized for correctly predicting the path through the tree. Since the other three models have better perplexities than the random and Huffman tree HLBL models, I assume this is because the other three trees start off with better word clusterings than is learned by the HLBL model with random or Huffman trees. This is supported by the fact that the $Q$ word-clouds are still uniform looking and unclustered (with some exceptions), for the random and Huffman HLBL models. Additionally, the other three models have the benefit of learning the paths better.
\paragraph{}
The $R$ matrix holds the learned vector representations for the decision tree nodes. In Figure \ref{fig:Rcloud}, we see that the HLBL models with the random and Huffman trees seem to learn some semantic or syntactic patterns in their $R$ matrices, whereas the other three trees do not. The two examples for the random tree cluster together numbers and cluster together names. The Huffman tree seems to have learned verbs and prepositions. In the first example for the Huffman tree, it looks like the left word is a present tense verb, and the right word is a past tense verb, with the exception of 'pressing'. Since the $Q$ and $R$ matrices are learned jointly, it makes sense that the $R$ matrices also show the same semantic/syntactic emphasis in learning. In comparison, the Brown Cluster tree and the Recursive Adaptive trees have round $R$ word-clouds. This means that the $R$ matrices for these three trees are close to multi-dimensional Gaussians, and do not help much with the models. It seems that the HLBL model does not learn $R$ representations when given trees that cluster semantically or syntactically words together. This seems to imply that most of the model parameters are captured in the $Q, C$ and $B$ matrices for HLBL models with such trees.

\begin{figure}[p]
\centering
\begin{tabular}{@{}m{1cm}m{1cm}m{3.5cm}m{3.5cm}m{3.5cm}@{}}
HLBL & $T_1$ &
\includegraphics[width=0.15\textheight]{./images/tsne/Q_random_it71_thumb.png} &
\includegraphics[width=0.15\textheight]{./images/tsne/Q_random_it71_small1.png} &
\includegraphics[width=0.15\textheight]{./images/tsne/Q_random_it71_small2.png}
\\
HLBL & $T_2$ &
\includegraphics[width=0.15\textheight]{./images/tsne/Q_Huff_iter132_thumb.png} &
\includegraphics[width=0.15\textheight]{./images/tsne/Q_Huff_iter132_small1.png} &
\includegraphics[width=0.15\textheight]{./images/tsne/Q_Huff_iter132_small2.png}
\\
HLBL &$T_3$ &
\includegraphics[width=0.15\textheight]{./images/tsne/Q_Brown_iter_118_thumb.png} &
\includegraphics[width=0.15\textheight]{./images/tsne/Q_Brown_iter_118_small1.png} &
\includegraphics[width=0.15\textheight]{./images/tsne/Q_Brown_iter_118_small2.png}
\\
HLBL & $T_4$ &
\includegraphics[width=0.15\textheight]{./images/tsne/Q_adaptiveR3_thumb.png} &
\includegraphics[width=0.15\textheight]{./images/tsne/Q_adaptiveR3_small1.png} &
\includegraphics[width=0.15\textheight]{./images/tsne/Q_adaptiveR3_small2.png}
\\
HLBL & $T_5$ &
\includegraphics[width=0.15\textheight]{./images/tsne/Q_adaptive-huffR1_it114_thumb.png} &
\includegraphics[width=0.15\textheight]{./images/tsne/Q_adaptive-huffR1_it114_small1.png} &
\includegraphics[width=0.15\textheight]{./images/tsne/Q_adaptive-huffR1_it114_small2.png}
\\
Factored LBL model &&
\includegraphics[width=0.15\textheight]{./images/tsne/Q_factored_it7_step0_05_thumb.png} &
\includegraphics[width=0.15\textheight]{./images/tsne/Q_factored_it7_step0_05_small1.png} &
\includegraphics[width=0.15\textheight]{./images/tsne/Q_factored_it7_step0_05_small2.png}
\end{tabular}
\caption{$Q$ matrices for various trees projected into two dimensions. The left-most image shows the general shape of all of the words together, and the center and right-most image show close-ups of the word-cloud.}
\label{fig:Qcloud}
\end{figure}

\begin{figure}[p]
\centering
\begin{tabular}{@{}m{1cm}m{1cm}m{3.5cm}m{3.5cm}m{3.5cm}@{}}

HLBL & $T_1$ &
\includegraphics[width=0.15\textheight]{./images/tsne/R_random_it71_thumb.png} &
\includegraphics[width=0.15\textheight]{./images/tsne/R_random_it71_small1.png} &
\includegraphics[width=0.15\textheight]{./images/tsne/R_random_it71_small2.png}
\\
HLBL & $T_2$ &
\includegraphics[width=0.15\textheight]{./images/tsne/R_Huff_iter132_thumb.png} &
\includegraphics[width=0.15\textheight]{./images/tsne/R_Huff_iter132_small1.png} &
\includegraphics[width=0.15\textheight]{./images/tsne/R_Huff_iter132_small2.png}
\\
HLBL & $T_3$ &
\includegraphics[width=0.15\textheight]{./images/tsne/R_Brown_iter_118_thumb.png} &
\includegraphics[width=0.15\textheight]{./images/tsne/R_Brown_iter_118_small1.png} &
\includegraphics[width=0.15\textheight]{./images/tsne/R_Brown_iter_118_small2.png}
\\
HLBL & $T_4$ &
\includegraphics[width=0.15\textheight]{./images/tsne/R_adaptiveR3_thumb.png} &
\includegraphics[width=0.15\textheight]{./images/tsne/R_adaptiveR3_small1.png} &
\includegraphics[width=0.15\textheight]{./images/tsne/R_adaptiveR3_small2.png}
\\
HLBL &$T_5$ &
\includegraphics[width=0.15\textheight]{./images/tsne/R_adaptive-huffR1_it114_thumb.png} &
\includegraphics[width=0.15\textheight]{./images/tsne/R_adaptive-huffR1_it114_small1.png} &
\includegraphics[width=0.15\textheight]{./images/tsne/R_adaptive-huffR1_it114_small2.png}
\\
Factored LBL model &  &
\includegraphics[width=0.15\textheight]{./images/tsne/R_factored_it7_step0_05_thumb.png} &
\includegraphics[width=0.15\textheight]{./images/tsne/R_factored_it7_step0_05_small1.png} &
\includegraphics[width=0.15\textheight]{./images/tsne/R_factored_it7_step0_05_small2.png}
\end{tabular}
\caption{$R$ matrices for various trees projected into two dimensions. The left-most image shows the general shape of all of the words together, and the center and right-most image show close-ups of the word-cloud.}
\label{fig:Rcloud}
\end{figure}

\subsection{word2vec $Q$ Initialization} \label{sec:ADAPTIVEQ}
\paragraph{}
Another experiment I tried was using better word representations to initialize $Q$. After observing that the HLBL model with the Brown Cluster tree and the Recursive ADAPTIVE trees do not seem to learn semantically or lexically motivated clusters in their $Q$ matrices, I thought that by improving the initialization of the $Q$ matrices, I could push the models to learn better $Q$ matrices. 
\paragraph{}
I use the word2vec toolkit representations implemented by Mikolov et al., to initialize my $Q$ matrix. The word2vec toolkit is based off of the ideas in \cite{Mikolov2013}. I create word2vec representations with a skip-gram model, the hierarchical soft-max, a word-window of 5, down-sampling of .001 and a representation dimensionality of 100. The representation creation takes 2.309 seconds to run with 12 threads on the WSJ dataset.  I test the new $Q$ initialization with the HLBL model using a Brown Cluster Tree, an ADAPTIVE tree, and also a Huffman tree. The results for the HLBL model with word2vec $Q$ initialization are in Table \ref{tab:brownWord2vec}.

\paragraph{}
The $Q$ initialization with the word2vec representations improves the perplexity of the HLBL model with a Brown Cluster tree~($T_3$) by 2.79\% and the entropy by 0.61\%. We also see that with better $Q$ initialization, the HLBL model with a Brown Cluster tree performs almost as well as the LBL model. The perplexity of the Brown Cluster tree is only -0.43 away from the LBL's perplexity on the WSJ dataset. Unfortunately, this still suffers from the long creation time of the Brown Cluster tree.
\paragraph{}
To avoid the long creation time of the Brown Cluster tree, I also tried using an ADAPTIVE tree~($T_6$). The ADAPTIVE tree tries to cluster similar word representations together in the tree. If the word representations capture semantic and syntactical information, then the ADAPTIVE tree acts much like the Brown Cluster tree. To keep the tree creation time low, I did not run the ADAPTIVE algorithm recursively. Using the same word2vec representations described above, I create an ADAPTIVE tree, and also initialize the $Q$ matrix to the word2vec representations. The ADAPTIVE tree does not perform as well as the Brown Cluster tree, but is faster to create. The ADAPTIVE tree is also faster overall than the LBL model but not the Factored LBL on the WSJ dataset.

\paragraph{}
I also tried initializing the $Q$ matrix with the word2vec representations for the Huffman tree~($T_2$) HLBL model with no performance gain. The Huffman tree model does a good job learning the context words even without sensible initialization. 

\begin{table*} \centering
\ra{1.3}
\begin{tabular}{cccccc}\toprule
Tree & Perplexity & Epochs & Training Time Per Epoch & Testing Time & Total Time\\ 
\midrule
$T_2$ & 115.786 & 81 & 26.31s &0.46s & 2133.98s \\
$T_3$ & 97.64 & 115& 30.92s & 0.49s& 275022.82s\\
$T_6$& 104.477 & 83& 35.46s & 0.56s& 5013.57s\\
\bottomrule
\end{tabular}
\caption{HLBL model with $Q$ initialized to word2vec representations on WSJ dataset.}
\label{tab:brownWord2vec}
\end{table*}

\subsection{Visualization of HLBL Models with word2vec $Q$ Initialization}
\paragraph{}
I ran the t-SNE visualizations described in Section \ref{sec:tsne} over the HLBL models with $Q$ initialized to the word2vec representations. Note that I did not include the Huffman tree with word2vec $Q$ initialization since the model does not seem to be better than its non-word2vec initialized counterpart. The results can be seen in Figures \ref{fig:QcloudWord2Vec} and \ref{fig:RcloudWord2Vec}. We see that the $Q$ word-clouds make more semantically and syntactically similar clusters unlike the models without the word2vec $Q$ initialization. This is exactly the result we were looking for. The $R$ word-clouds remain largely unchanged. 

\begin{figure}[p]
\centering
\begin{tabular}{@{}m{2cm}m{3.5cm}m{3.5cm}m{3.5cm}@{}}
$T_3$&
\includegraphics[width=0.15\textheight]{./images/tsne/Q_Brown_QinWord2Vec_thumb.png} &
\includegraphics[width=0.15\textheight]{./images/tsne/Q_Brown_QinWord2Vec_small1.png} &
\includegraphics[width=0.15\textheight]{./images/tsne/Q_Brown_QinWord2Vec_small2.png}
\\
$T_6$ &
\includegraphics[width=0.15\textheight]{./images/tsne/Q_adaptive_QinWord2Vec_RinWord2Vec_thumb.png} &
\includegraphics[width=0.15\textheight]{./images/tsne/Q_adaptive_QinWord2Vec_RinWord2Vec_small1.png} &
\includegraphics[width=0.15\textheight]{./images/tsne/Q_adaptive_QinWord2Vec_RinWord2Vec_small2.png}
\end{tabular}
\caption{$Q$ matrices for HLBL models with $Q$ initialized to word2vec representations. The left-most image shows the general shape of all of the words together, and the center and right-most image show close-ups of the word-cloud.}
\label{fig:QcloudWord2Vec}
\end{figure}

\begin{figure}[p]
\centering
\begin{tabular}{@{}m{2cm}m{3.5cm}m{3.5cm}m{3.5cm}@{}}
$T_2$ &
\includegraphics[width=0.15\textheight]{./images/tsne/R_Brown_QinWord2Vec_thumb.png} &
\includegraphics[width=0.15\textheight]{./images/tsne/R_Brown_QinWord2Vec_small1.png} &
\includegraphics[width=0.15\textheight]{./images/tsne/R_Brown_QinWord2Vec_small2.png}
\\
$T_6$ &
\includegraphics[width=0.15\textheight]{./images/tsne/R_adaptive_QinWord2Vec_RinWord2Vec_thumb.png} &
\includegraphics[width=0.15\textheight]{./images/tsne/R_adaptive_QinWord2Vec_RinWord2Vec_small1.png} &
\includegraphics[width=0.15\textheight]{./images/tsne/R_adaptive_QinWord2Vec_RinWord2Vec_small2.png}
\end{tabular}
\caption{$R$ matrices for HLBL models with $Q$ initialized to word2vec representations. The left-most image shows the general shape of all of the words together, and the center and right-most image show close-ups of the word-cloud.}
\label{fig:RcloudWord2Vec}
\end{figure}


\subsection{Large Vocabulary}
\paragraph{}
Real languages have much larger vocabularies than the WSJ dataset. The Oxford English Dictionary currently has 171,476 full word entries for English \cite{OED}. For a morphologically rich language, the vocabulary size can easily exceed that of English.
\paragraph{}
The HLBL model really shines in terms of speed with large vocabulary sizes. The speedup using the HLBL model is order $O(\frac{|V|}{\log|V|})$. To highlight this, I also run some experiments with a subset of the British National Corpus (BNC). I use a random 120,000 lines of the BNC, where 100,000 lines are randomly assigned to the training set and the other 20,000 lines are assigned to the test set. Due to an error in generating the test set, 80,000 of the training set sentences are also in the test set. This simply means the testing perplexities are better than they really should be for all models. Since the BNC dataset is only being used to highlight the speed difference of models, I did not rerun the experiments. The duplication of training examples in the BNC test set makes the test set larger, and actually helps highlight the testing speed difference of the models. There are 2,141,639 training tokens and 2,138,171 testing tokens. The training set has a vocabulary size of 84,439 (including "\_UNK\_"). The test set is preprocessed to replace any word not seen in the training set with "\_UNK\_". We can see on the 85K word vocabulary where non-HLBL models slow down immensely. Note that while we can see slow-downs with an 85K vocabulary, the BNC dataset is still very small in comparison to data used in commercial applications. An industrial language model will use corpora with hundreds of thousands of words in their vocabulary and billions or trillions of training tokens. The training time will generally dominate the tree creation time on such large corpora.

\paragraph{}
Using the 85K vocabulary BNC dataset, I run the HLBL model with two different trees, the Huffman tree and also the ADAPTIVE(word2vec) tree with Q initialized to the word2vec representations. The BNC word2vec representations take 4.960s to create running with 12 threads. The results are shown in Table \ref{tab:largeVocabulary}.
\paragraph{}
The results show why a hierarchical speed up technique is useful for computing language models over large corpora. The Factored LBL model that performs well in terms of perplexity and speed on the 10K WSJ vocabulary, performs much slower with the 85K BNC vocabulary. The Factored LBL model's training and testing times scale linearly with the vocabulary size. The HLBL model on the other hand scales logarithmically.  We see the HLBL models are faster both in training and testing than the Factored LBL model. The HLBL model with the Huffman tree is $4.46\times$ faster to compute than the Factored LBL model. Though there is a perplexity loss of $20.72\%$ and an entropy loss of $4.46\%$. The HLBL model with the ADAPTIVE(word2vec) tree and $Q$ initialized to word2vec representations is $3.89\times$ faster to compute than the Factored LBL model and has a perplexity loss of $9.34\%$ and an entropy loss of $1.93\%$ This makes the HLBL model with the ADAPTIVE tree and word2vec $Q$ initialization the best of the HLBL models I explored.

%\begin{table*} \centering
%\ra{1.3}
%\begin{tabular}{@{}ccC{2cm}cc@{}}\toprule
%Model & Perplexity & Training Time Per Epoch & Testing Time & Total Time\\ 
%\midrule
% LBL&105.442 &111.05s & 21.47s & 6906.57s \\
% Factored LBL &108.802& 71.05s & 0.60s & 7887.15s\\
% HLBL with Huffman Tree &131.913& 33.26s &0.48s &3492.79s \\
%\bottomrule
%\end{tabular}
%\caption{The effect of large vocabularies on training and testing times}
%\label{tab:largeVocabulary}
%\end{table*}

\begin{table*} \centering
\ra{1.3}
\begin{tabular}{@{}C{2cm}ccccC{2cm}C{2cm}C{2cm}@{}}\toprule
Model & Tree & Q init & Perplexity & Epochs & Training Time Per Epoch & Testing Time & Total Time\\ 
\midrule
 Factored LBL & - & Gaussian &144.32& 112 & 479.37s & 72.14s & 53761.58s\\
 HLBL& $T_2$ & Gaussian &182.06& 117& 102.7s &30.18s &12047.61s \\
 HLBL & $T_6$ & word2vec &159.187& 106& 129.81s &38.32s &13803.14s \\
\bottomrule
\end{tabular}
\caption{The effect of a large vocabulary (85K) from the BNC dataset on training and testing times}
\label{tab:largeVocabulary}
\end{table*}
  
  
  


%now enable appendix numbering format and include any appendices
\appendix

\chapter{Derivation of HLBL Gradients}
\paragraph{}
We would like to find the gradient to the negative log likelihood with an $L_2$ regularizer with respect to its parameters $\theta$.

Given:
\begin{align*}
&-\ln(J(\theta;w_1,\dots, w_m)) = \sum_{i=1}^{m} \sum_j -\ln(P(d_j(w_i) | q_j, w_{i-(n-1)},\dots, w_{i-1}))
\\ 
&-\ln(P(d_j(w_i) | q_j, w_{i-(n-1)},\dots, w_{i-1})) =  
\begin{cases}
  \ln (1 + e^{-\hat{r}^T q_{j} -b_{j}}) & \text{if } d_j(w_i)  = 1 \\
  \ln (1 + e^{\hat{r}^T q_{j} +b_{j}})     & \text{if } d_j(w_i) = 0
  \end{cases}
\\
&P(d_j(w_i) | q_j, w_{i-(n-1)},\dots, w_{i-1}) =  
\begin{cases}
  \sigma(\hat{r}^T q_{j} +b_{j}) & \text{if } d_j(w_i)  = 1 \\
  \sigma(-\hat{r}^T q_{j} -b_{j})     & \text{if } d_j(w_i) = 0
  \end{cases}
\\
&\hat{r} = \sum_{i=1}^{n-1} C_i r_{w_i} 
\\
&\sigma(x) = \frac{1}{1+e^{-x}}
\\
&\sigma(x) +\sigma(-x) = 1
\end{align*}

We will solve for the gradient:
\begin{align*}
&\frac{\partial}{\partial \theta} -\ln(J(\theta;w_1,\dots, w_m)) +  \mu ||\theta||^2_2
\\
& \sum_{i=1}^{m} \sum_j \frac{\partial}{\partial \theta} -\ln(P(d_j(w_i) | q_j, w_{i-(n-1)},\dots, w_{i-1})) +\frac{\partial}{\partial \theta}  \mu ||\theta||^2_2
\\
& \sum_{i=1}^{m} \sum_j \frac{\partial}{\partial \theta} -\ln(P(d_j(w_i) | q_j, w_{i-(n-1)},\dots, w_{i-1})) + 2 \mu \theta
\end{align*}

When $d_j(w_i)  = 1$:
\begin{align*}
& \frac{\partial}{\partial \theta} -\ln(P(d_j(w_i) | q_j, w_{i-(n-1)},\dots, w_{i-1})) 
\end{align*}
\begin{align*}
&=  \frac{\partial}{\partial \theta} \ln (1 + e^{-\hat{r}^T q_{j} -b_{j}})
\\
&= \frac{1}{1 + e^{-\hat{r}^T q_{j} -b_{j}}} \frac{\partial}{\partial \theta}1 + e^{-\hat{r}^T q_{j} -b_{j}}
\\
&= \frac{e^{-\hat{r}^T q_{j} -b_{j}}}{1 + e^{-\hat{r}^T q_{j} -b_{j}}} \frac{\partial}{\partial \theta}-\hat{r}^T q_{j} -b_{j}
\\
&= \frac{-e^{-\hat{r}^T q_{j} -b_{j}}}{1 + e^{-\hat{r}^T q_{j} -b_{j}}} \frac{\partial}{\partial \theta}\hat{r}^T q_{j} +b_{j}
\\
&= \left( \frac{1}{1 + e^{-\hat{r}^T q_{j} -b_{j}}}  - \frac{1 + e^{-\hat{r}^T q_{j} -b_{j}}}{1 + e^{-\hat{r}^T q_{j} -b_{j}}} \right) \frac{\partial}{\partial \theta}\hat{r}^T q_{j} +b_{j}
\\
&= \left( P(d_j(w_i)=1 | q_j, w_{i-(n-1)},\dots, w_{i-1}) - 1 \right) \frac{\partial}{\partial \theta}\hat{r}^T q_{j} +b_{j}
\end{align*}

When $d_j(w_i)  = 0$:
\begin{align*}
& \frac{\partial}{\partial \theta} -\ln(P(d_j(w_i) | q_j, w_{i-(n-1)},\dots, w_{i-1})) 
\end{align*}
\begin{align*}
&=  \frac{\partial}{\partial \theta} \ln (1 + e^{\hat{r}^T q_{j} +b_{j}})
\\
&= \frac{1}{1 + e^{\hat{r}^T q_{j} +b_{j}}} \frac{\partial}{\partial \theta}1 + e^{\hat{r}^T q_{j} +b_{j}}
\\
&= \frac{e^{\hat{r}^T q_{j} +b_{j}}}{1 + e^{\hat{r}^T q_{j} +b_{j}}} \frac{\partial}{\partial \theta}\hat{r}^T q_{j} +b_{j}
\\
&= \frac{1}{1 + e^{-\hat{r}^T q_{j} -b_{j}}}  \frac{\partial}{\partial \theta}\hat{r}^T q_{j} +b_{j}
\\
&= \left( P(d_j(w_i)=1 | q_j, w_{i-(n-1)},\dots, w_{i-1}) - 0 \right) \frac{\partial}{\partial \theta}\hat{r}^T q_{j} +b_{j}
\end{align*}

$\frac{\partial}{\partial \theta}\hat{r}^T q_{j} +b_{j}$ with respect to $b_j$:
\begin{align*}
\frac{\partial}{\partial r_j}\hat{r}^T q_{j} +b_{j} = 1
\end{align*}

$\frac{\partial}{\partial \theta}\hat{r}^T q_{j} +b_{j}$ with respect to $q_j$:
\begin{align*}
\frac{\partial}{\partial r_j}\hat{r}^T q_{j} +b_{j} = \hat{r}
\end{align*}

$\frac{\partial}{\partial \theta}\hat{r}^T q_{j} +b_{j}$ with respect to $r_{w_i}$:
\begin{align*}
\frac{\partial}{\partial r_{w_i}}(\sum_{i=1}^{n-1} C_i r_{w_i})^T q_{j} +b_{j} = C_i^T q_j
\end{align*}

$\frac{\partial}{\partial \theta}\hat{r}^T q_{j} +b_{j}$ with respect to $C_i$:
\begin{align*}
\frac{\partial}{\partial C_i}(\sum_{i=1}^{n-1} C_i r_{w_i})^T q_{j} +b_{j} = r_{w_i} q_j^T
\end{align*}



\chapter{Visualization of the $R$ and $Q$ matrices}
\paragraph{}
To get a better idea of the types of representations learned by the models, I present the t-SNE visualizations of the learned $R$ and $Q$ matrices for the HLBL model with a Brown Cluster tree, and the Factored LBL model. I feel the Brown Cluster HLBL model has visualizations that are representative of the other HLBL models. The Factored LBL model visualizations are provided as a base-line. Only the first 1000 vectors of each matrix are shown.

%start of Q
%\begin{figure}[h]
%\includegraphics[height=.85\textheight,trim=100 0 0 0]{./images/tsne/1000s/Q_random_it71_1000words.png} 
%\centering
%\caption{First 1000 vectors of the Q matrix for the HLBL model with a random tree projected into two-dimensions using t-SNE}
%\end{figure}

%\begin{figure}[h]
%\includegraphics[height=.85\textheight,trim=100 0 0 0]{./images/tsne/1000s/Q_Huff_iter132_1000words.png} 
%\centering
%\caption{First 1000 vectors of the Q matrix for the HLBL model with a Huffman tree projected into two-dimensions using t-SNE}
%\end{figure}

\begin{figure}[ht]
\includegraphics[height=.85\textheight,trim=100 0 0 0]{./images/tsne/1000s/Q_Brown_iter_118_1000words.png} 
\centering
\caption{First 1000 vectors of the $Q$ matrix for the HLBL model with a Brown Cluster tree projected into two-dimensions using t-SNE.}
\end{figure}
%
%\begin{figure}[h]
%\includegraphics[height=.85\textheight,trim=100 0 0 0]{./images/tsne/1000s/Q_adaptiveR3_1000words.png} 
%\centering
%\caption{First 1000 vectors of the Q matrix for the HLBL model with an ADAPTIVE(3,random) tree projected into two-dimensions using t-SNE}
%\end{figure}
%
%\begin{figure}[h]
%\includegraphics[height=.85\textheight,trim=100 0 0 0]{./images/tsne/1000s/Q_adaptive-huffR1_it114_1000words.png} 
%\centering
%\caption{First 1000 vectors of the Q matrix for the HLBL model with an ADAPTIVE(2,Huffman) tree projected into two-dimensions using t-SNE}
%\end{figure}
%
%
%\begin{figure}[h]
%\includegraphics[height=.85\textheight,trim=100 0 0 0]{./images/tsne/1000s/Q_Brown_QinWord2Vec_1000words.png} 
%\centering
%\caption{First 1000 vectors of the Q matrix for the HLBL model with a Brown Cluster tree and word2vec Q initialization projected into two-dimensions using t-SNE}
%\end{figure}

%\begin{figure}[h]
%\includegraphics[height=.85\textheight,trim=100 0 0 0]{./images/tsne/1000s/Q_adaptive_QinWord2Vec_RinWord2Vec_1000words.png} 
%\centering
%\caption{First 1000 vectors of the Q matrix for the HLBL model with a ADAPTIVE(word2vec) tree and word2vec Q initialization projected into two-dimensions using t-SNE}
%\end{figure}

\begin{figure}[ht]
\includegraphics[height=.85\textheight,trim=100 0 0 0]{./images/tsne/1000s/Q_Factored_it7_step0_05_1000words.png} 
\centering
\caption{First 1000 vectors of the $Q$ matrix for the Factored LBL model projected into two-dimensions using t-SNE.}
\end{figure}



%start of R

%\begin{figure}[h]
%\includegraphics[height=.85\textheight,trim=120 0 0 0]{./images/tsne/1000s/R_random_it71_1000words.png} 
%\centering
%\caption{First 1000 vectors of the R matrix for the HLBL model with a random tree projected into two-dimensions using t-SNE}
%\end{figure}

%\begin{figure}[h]
%\includegraphics[height=.85\textheight,trim=120 0 0 0]{./images/tsne/1000s/R_Huff_iter132_1000words.png} 
%\centering
%\caption{First 1000 vectors of the R matrix for the HLBL model with a Huffman tree projected into two-dimensions using t-SNE}
%\end{figure}

\begin{figure}[ht]
\includegraphics[height=.85\textheight,trim=120 0 0 0]{./images/tsne/1000s/R_Brown_iter_118_1000words.png} 
\centering
\caption{First 1000 vectors of the $R$ matrix for the HLBL model with a Brown Cluster tree projected into two-dimensions using t-SNE.}
\end{figure}
%
%\begin{figure}[h]
%\includegraphics[height=.85\textheight,trim=120 0 0 0]{./images/tsne/1000s/R_adaptiveR3_1000words.png} 
%\centering
%\caption{First 1000 vectors of the R matrix for the HLBL model with an ADAPTIVE(3,random) tree projected into two-dimensions using t-SNE}
%\end{figure}
%
%\begin{figure}[h]
%\includegraphics[height=.85\textheight,trim=120 0 0 0]{./images/tsne/1000s/R_adaptive-huffR1_it114_1000words.png} 
%\centering
%\caption{First 1000 vectors of the R matrix for the HLBL model with an ADAPTIVE(2,Huffman) tree projected into two-dimensions using t-SNE}
%\end{figure}
%
%
%\begin{figure}[h]
%\includegraphics[height=.85\textheight,trim=120 0 0 0]{./images/tsne/1000s/R_Brown_QinWord2Vec_1000words.png} 
%\centering
%\caption{First 1000 vectors of the R matrix for the HLBL model with a Brown Cluster tree and word2vec Q initialization projected into two-dimensions using t-SNE}
%\end{figure}

%\begin{figure}[h]
%\includegraphics[height=.85\textheight,trim=120 0 0 0]{./images/tsne/1000s/R_adaptive_QinWord2Vec_RinWord2Vec_1000words.png} 
%\centering
%\caption{First 1000 vectors of the R matrix for the HLBL model with a ADAPTIVE(word2vec) tree and word2vec Q initialization projected into two-dimensions using t-SNE}
%\end{figure}

\begin{figure}[ht]
\includegraphics[height=.85\textheight,trim=120 0 0 0]{./images/tsne/1000s/R_Factored_it7_step0_05_1000words.png} 
\centering
\caption{First 1000 vectors of the $R$ matrix for the Factored LBL model projected into two-dimensions using t-SNE.}
\end{figure}


\chapter{HLBL Implementation}
\lstinputlisting[language=C++]{../oxlm/lbl/train_hlbl.cc}



%next line adds the Bibliography to the contents page
\addcontentsline{toc}{chapter}{References}
%uncomment next line to change bibliography name to references
\renewcommand{\bibname}{References}
\bibliography{refs}        %use a bibtex bibliography file refs.bib
\bibliographystyle{plain}  %use the plain bibliography style

\end{document}

